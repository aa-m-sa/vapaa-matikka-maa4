\chapter{Esitietoja}
	\section{Itseisarvo}


\laatikko{
KIRJOITA TÄHÄN LUKUUN

\begin{itemize}
\item itseisarvon määritelmä (paloittain määritelty muoto)
\item geometrinen tulkinta lukusuoralla
\item itseisarvoyhtälöistä tyyppiä $|f(x)|=a$, $|f(x)|=|g(x)|$,
$|f(x)|=g(x)$
\end{itemize}

KIITOS!}


\qrlinkki{http://opetus.tv/maa/maa4/itseisarvo/}{Opetus.tv: Itseisarvolauseke ja itseisarvon ominaisuuksia}


% Rumasti toteutettu, mutta toistaiseksi toimii.
% Korjatkaa, jos joku saa siistimmin aikaiseksi.
\begin{lukusuora}{-5}{5}{10}
	\lukusuorapiste{0}{$0$}
	\lukusuorapiste{3}{}
	\lukusuorapiste{-3}{}
	\lukusuoraalanimi{3.1}{$a$}
	\lukusuoraalanimi{-3.3}{$-a$}
	\lukusuoranimi{1.5}{$\overbrace{\hspace{27 mm}}^{|a|}$}
	\lukusuoranimi{-1.5}{$\overbrace{\hspace{27 mm}}^{|-a|}$}
	
	\lukusuorapiste{2}{}
	\lukusuorapiste{-2}{}
	\lukusuoraalanimi{-2.3}{$-2$}
	\lukusuoraalanimi{2.1}{$2$}
	\lukusuoraalanimi{1}{$\underbrace{\hspace{18 mm}}_{|2|=2}$}
	\lukusuoraalanimi{-1}{$\underbrace{\hspace{18 mm}}_{|-2|=2}$}
\end{lukusuora}


Itseisarvo voidaan tulkita lukusuoralla pisteen etäisyytenä nollasta.
Epänegatiivisen luvun itseisarvo on luku itse ja
negatiivisen luvun itseisarvo on luvun vastaluku.

\laatikko{
	\textbf{Itseisarvon määritelmä}\\
	$|x|= \begin{cases}
			x, & \kun x \geq 0 \\
			-x, & \kun x < 0
	\end{cases}$
}

Itseisarvo on funktio
\[
||: \mathbb{R} \rightarrow \mathbb{R}, \;
|x| = \begin{cases}
		x, & \kun x \geq 0 \\
		-x, & \kun x < 0.
	\end{cases}
\]

\begin{esimerkki}
Esitä ilman itseisarvomerkkejä
	\begin{alakohdat}
		\alakohta{$|3-\pi|$}
		\alakohta{$|x-3|$}
	\end{alakohdat}
	\textbf{Ratkaisu}
	\begin{alakohdat}
		\alakohta{Koska $3-\pi\approx-0,14<0$, niin $|3-\pi|=-(3-\pi)=\pi-3$}
		\alakohta{Koska $x-3\geq 0$, kun $x\geq3$, niin\\
			$|x-3|= \begin{cases}
				x-3, & \kun x \geq 3 \\
				-x+3, & \kun x < 3
			\end{cases}$
		}
	\end{alakohdat}
\end{esimerkki}

\vspace{10 mm}

\laatikko{
\textbf{Itseisarvon ominaisuuksia}\\ \\
	\begin{tabular}{l l}
		$|a|\geq0$ & Itseisarvo on aina epänegatiivinen \\
		$|a|=|-a|$ & Luvun ja sen vastaluvun itseisarvot ovat yhtäsuuret \\
		$|a|^2=a^2$ & Luvun itseisarvon neliö on yhtäsuuri kuin luvun neliö \\
		$|ab|=|a||b|$ & Tulon itseisarvo \\
		$\Bigl|\dfrac{a}{b}\Bigr|=\dfrac{|a|}{|b|}$ & Osamäärän itseisarvo

	\end{tabular}
}

\begin{esimerkki}
	Esitä lauseke $|x^2-16|+3x$ ilman itseisarvomerkkejä.\\
	\textbf{Ratkaisu} \\
	Tutkitaan ensin lausekkeen $x^2-16$ merkit.
	
	\begin{align*}
		\text{Nollakohdat: } \qquad	x^2-16 &=0 \\
			x^2 &=16 \\
			x &=\pm\sqrt{16} \\
			x &=\pm 4
	\end{align*}
	
	Lausekkeen $x^2-16$ kuvaaja on ylöspäin aukeneva paraabeli, joka leikkaa x-akselin kohdissa $-4$ ja $4$.
	
% Käyttö:
% \begin{lukusuora}{-1}{1}{10}
% \lukusuoraparaabeli{0.2}{0.8}{1}
% \lukusuoravalisa{0.2}{0.8}{x}{y}
% \lukusuoravaliaa{-0.5}{-0.3}{a}{b}
% \lukusuorapiste{0}{$x^2+y^2$}
% \lukusuoranimi{0.5}{$+$}
% \lukusuoraalanimi{0.05}{$-$}
% \end{lukusuora}
% 
% luo lukusuoran joka piirretään 10 pituisena, käyttää sisäisesti
% koordinaatteja -1..1, ja jolla näytetään väli [0.2, 0.8[ ja
% väli ]-0.5, -0.3[ nimillä [x, y[ ja ]a, b[. Näyttää pisteen
% 0 nimellä $x^2+y^2$ ja piirtää paraabelin jonka nollakohdat ovat
% 0.2 ja 0.8 ja huippu 1. Lisäksi se laittaa +-merkin suoran yläpuolelle
% kohtaan 0.5 ja --merkin suoran alapuolelle kohtaan 0.05.
%
% Mielivaltaisia kuvaajia saa lukusuoralle lukusuorakuvaaja-komennolla.
% Nuolia lukujen välille saa lukusuoranuoli-komennolla.


\begin{lukusuora}{-10}{10}{8}
	\lukusuoraparaabeli{-4}{4}{-1.5}
	\lukusuorapiste{-4}{$-4$}
	\lukusuorapiste{4}{$4$}
	\lukusuoranimi{-6}{$+$}
	\lukusuoranimi{6}{$+$}
	\lukusuoraalanimi{0}{$-$}
\end{lukusuora}

Kun $-4<x<4$, niin $x^2-16<0$. Tällöin
\[
|x^2-16|+3x = -(x^2-16)+3x = -x^2+16+3x=-x^2+3x+16.
\]
Kun $x\leq-4$ tai $x\geq4$, on $x^2-16\geq0$, ja tällöin
\[
|x^2-16|+3x =x^2-16+3x=x^2+3x-16
\]

Siis:\\

			$|x^2-16|+3x=
			\begin{cases}
				-x^2+3x+16, & \kun -4<x<4 \\
				x^2+3x-16, & \kun x\leq -4 \tai x\geq 4
			\end{cases}$

\end{esimerkki}

\newpage
\subsection*{Itseisarvoyhtälö $|f(x)|=a$}

\qrlinkki{http://opetus.tv/maa/maa4/itseisarvoyhtalo/}{Opetus.tv: Itseisarvoyhtälö}

Itseisarvoyhtälössä muuttuja on itseisarvomerkkien sisällä. Itseisarvoyhtälö voidaan ratkaista esittämällä yhtälö ilman itseisarvo merkkejä hyödyntäen itseisarvon määritelmää ja ominaisuuksia, ja ratkaisemalla saadut yhtälöt.

\laatikko{
\textbf{Yhtälö: \quad $|f(x)|=a$}\\
Jos $a\geq0$, niin
\[
	f(x)=a  \quad \tai \quad f(x)=-a
\]
Jos $a<0$, niin yhtälöllä $|f(x)|=a$ ei ole ratkaisuja.

}

\begin{esimerkki}

Ratkaise yhtälö
	\begin{alakohdat}
		\alakohta{ $|x|=3$}
		\alakohta{ $|x|=-2$}
	\end{alakohdat}
	\textbf{Ratkaisu} \\
	\begin{alakohdat}
		\alakohta{Ainoastaan lukujen $3$ ja $-3$ itseisarvot ovat 3, joten $x=3$ tai $x=-3$. Siis $x=\pm 3$}
		\alakohta{Itseisarvo ei voi olla negatiivinen, joten yhtälöllä ei ole ratkaisua.}
	\end{alakohdat}

\end{esimerkki}


\begin{esimerkki}
Ratkaise yhtälö $|3x-4|=2$. \\
\textbf{Ratkaisu}\\
Yhtälön ratkaisut saadaan, kun luvun $3x-4$ etäisyys nollasta on 2, eli ainoastaan jos luku on $2$ tai $-2$. Saadaan:
	\begin{align*}
		3x-4&=2    &\tai  \qquad    \qquad      3x-4&=-2 \\
		3x&=6      &     \qquad      3x&= 2 \\
		x&=2       &     \qquad          x&=\frac{2}{3}
	\end{align*}
Ratkaisun oikeellisuuden voi tarkistaa sijoittamalla saadut ratkaisut alkupäiseen yhtälöön.
\[|3\cdot2-4|=|6-4|=|2|=2 \quad \text{ ja } \quad \vert3\cdot\dfrac{2}{3}-4\vert=|2-4|=|-2|=2\text{.}\]

\textbf{Vastaus: } $x=2$  tai  $x=\dfrac{2}{3}$.
\end{esimerkki}

\begin{esimerkki}
Ratkaise yhtälö $3x=3+|2x-3|$. (YO S82/1)\\
\textbf{Ratkaisu}\\
\begin{align*}
	3x&=3+|2x-3|  \\
	3x-3&=\underbrace{|2x-3|}_{\geq0}  \\
\end{align*}
Koska $|2x-3|\geq0$ aina, niin on oltava myös $3x-3\geq0$. Tällöin
\begin{align*}
	3x-3&\geq0 \\
	3x&\geq3 \\
	x&\geq1
\end{align*}
Itseisarvoyhtälöstä saadaan:
\begin{align*}
	2x-3&=3x-3   &\tai  \qquad \qquad 2x-3&=-(3x-3) \\
	-x&=0        &\ \                  2x-3&=-3x+3 \\
	x&=0         &\ \                  5x&=6 \\
	\text{ei kelpaa,}&\text{ oltava}x\geq1  &\ \  x&=\frac{6}{5} \\
	&\ \         &\ \                 &\text{kelpaa}
\end{align*}
\textbf{Vastaus: }$x=\dfrac{6}{5}=1\dfrac{1}{5}$.

\end{esimerkki}


Edellisen esimerkin tilannetta $3x-3=|2x-3|$ voidaan tarkastella myös piirtämällä kuvaajat $y=3x-3$ ja $y=|2x-3|$. Lausekkeet ovat yhtäsuuret, kun niiden saamat $y$:n arvot ovat yhtäsuuret, siis kohdissa, joissa kuvaajat leikkaavat toisensa.


\begin{kuva}
    kuvaaja.pohja(-1.5, 4.5, -2, 3, nimiX = "$x$", nimiY = "$y$")
    piste((1.2, 0.6), "",(-1,0))
    vari("red")
    kuvaaja.piirra("3*x-3", nimi = "$y=3x - 3$",suunta = (1, 0), kohta=0.7)
    vari("blue")
    kuvaaja.piirra("abs(2*x-3)", nimi = "$y=|2x - 3|$", kohta = 2.2)
\end{kuva}

Huomaamme, että lausekkeet saavat saman arvon kohdassa, jossa \mbox{$x=\frac{6}{5}$}. Esimerkkiä ratkaistaessa vastaan tullut epäkelpo ratkaisu $x=0$ olisi vastannut suoran $y=2x-3$ (ilman itseisarvo merkintää) leikkauspistettä suoran $y=3x-3$ kanssa, eli tilannetta, jossa sininen suora olisi kulkenut myös $x$-akselin alapuolella.

\subsection*{Yhtälö $|f(x)|=|g(x)|$}

Merkintä $|a|=|b|$ tarkoittaa, että luvut $a$ ja $b$ ovat lukusuoralla yhtä kaukana nollasta. Tällöin lukujen $a$ ja $b$ täytyy olla joko samat tai toistensa vastaluvut.

\laatikko{
$|f(x)|=|g(x)| \quad \ekvi \quad f(x)=g(x) \quad \tai \quad f(x)=-g(x)$.}

Tällaisen yhtälön voi ratkaista myös neliöön korottamalla ominaisuuden $|a|^2=a^2$ avulla. Jos yhtälön molemmat puolet ovat ei-negatiivisia, niin yhtälön yhtäsuuruus säilyy neliöön korotettaessa.

\laatikko {$|f(x)|=|g(x)| \quad \ekvi \quad f(x)^2=g(x)^2$.}

\begin{esimerkki}
	Ratkaise yhtälö $|2x-4|=|3-x|$ \\
	\textbf{Ratkaisu (tapa 1)} \\
	Lukujen $2x-4$ ja $3-x$ tulee olla yhtäsuuret tai toistensa vastaluvut.
	\begin{align*}
		|2x-4|&=|3-x|  &\quad   &\ \ \\
		&\quad \  &\quad \ &\quad \ \\
		2x-4&=3-x  &\tai \qquad \qquad 2x-4&=-(3-x) \\
		3x&=7 \quad   &\quad     2x-4&=-3+x \\
		x&=\frac{7}{3} \quad  &\quad    x&=1
	\end{align*}

	\textbf{Vastaus: } $x=\frac{7}{3}$ \quad tai \quad $x=1$

	\textbf{Ratkaisu (tapa 2)} \\
	Yhtälön molemmat puolet voidaan korottaa neliöön.
				
	\begin{align*}
		\underbrace{|2x-4|}_{\geq0} &= \underbrace{|3-x|}_{\geq0}    \\
		|2x-4|^2 &= |3-x|^2   \\
		(2x-4)^2 &= (3-x)^2   \\
		4x^2-16x+16 &= 9 -6x + x^2   \\
		3x^2-10x+7 &= 0   \\
		x &= \frac{10\pm\sqrt{(-10)^2-4\cdot 3\cdot 7}}{2\cdot 3}   \\
		x &= \frac{10\pm\sqrt{16}}{6}   \\		
		x &= \frac{10\pm4}{6}     \\
		x=\frac{14}{6}=\frac{7}{3} \quad  &\tai \quad x=\frac{6}{6}=1 \\
	\end{align*}
	\textbf{Vastaus: } $x=\frac{7}{3}$ \quad tai \quad $x=1$

\end{esimerkki} 

\begin{esimerkki}
	Ratkaise yhtälö $|2x-4|=|3-x|+2x$.
	
	\textbf{Ratkaisu} \\
	Huomataan, ettei edellisen esimerkin ratkaisutavat toimi, sillä yhtälö ei ole muotoa $|f(x)|=|g(x)|$. Neliöön korotuskaan ei toimi, sillä yhtälön oikea puoli voi saada negatiivisia arvoja. Ratkaistaan yhtälö poistamalla epäyhtälömerkit ja ratkaisemalla syntyvät yhtälöt alueittain. Ratkaisun apuna voidaan hyödyntää merkkikaaviota, johon merkitsemme itseisarvojen sisäisten lausekkeiden saamat merkit kullakin välillä.\\
	
	
\begin{lukusuora}{0}{4}{4}
	\lukusuorakuvaaja{2*x-4}
	\lukusuoranimi{1}{$2x-4$}
	\lukusuoraalanimi{1}{$-$}
	\lukusuoranimi{3}{$+$}
	\lukusuorapiste{2}{$2$}
\end{lukusuora}
\begin{lukusuora}{1}{5}{4}
	\lukusuorakuvaaja{3-x}
	\lukusuoraalanimi{2}{$3-x$}
	\lukusuoranimi{2}{$+$}
	\lukusuoraalanimi{4}{$-$}
	\lukusuorapiste{3}{$3$}
\end{lukusuora}

\begin{center}
\begin{merkkikaavio}{2}
\merkkikaavioKohta{$2$}
\merkkikaavioKohta{$3$}

	\merkkikaavioFunktio{$2x-4$}
	\merkkikaavioMerkki{$-$}
	\merkkikaavioMerkki{$+$}
	\merkkikaavioMerkki{$+$}

\merkkikaavioUusirivi
	\merkkikaavioFunktio{$3-x$}
	\merkkikaavioMerkki{$+$}
	\merkkikaavioMerkki{$+$}
	\merkkikaavioMerkki{$-$}

\end{merkkikaavio}
\end{center}

Ratkaistaan yhtälö alueittain\\
(1.) \quad Jos $x<2$, niin saadaan:
\begin{align*}
	|\underbrace{2x-4}_{<0}|&=|\underbrace{3-x}_{>0}|+2x \quad \Vert x<2  \\
	-(2x-4)&=(3-x)+2x \\
	-2x+4&=3-x+2x \\
	-3x &= -1 \\
	x &= \frac{1}{3} \quad \text{kelpaa}
\end{align*}

(2.) \quad Jos $2\leq x\leq 3$, niin saadaan:
\begin{align*}
	|\underbrace{2x-4}_{\geq0}|&=|\underbrace{3-x}_{\geq0}|+2x \quad \Vert 2\leq x\leq 3  \\
	(2x-4)&=(3-x)+2x \\
	2x-4&=3-x+2x \\
	x &= 7 \quad \text{ei kelpaa, sillä ei ole välillä $2\leq x\leq 3$.}
\end{align*}

(3.)\quad Jos $x>3$, niin saadaan:
\begin{align*}
	|\underbrace{2x-4}_{>0}|&=|\underbrace{3-x}_{<0}|+2x \quad \Vert x>3  \\
	(2x-4)&=-(3-x)+2x \\
	2x-4&=-3+x+2x \\
	-x &= 1 \\
	x &= -1  \quad \text{ei kelpaa, sillä ei ole välillä $x>3$.}
\end{align*}

\textbf{Vastaus: }\quad $x=\dfrac{1}{3}$.

\end{esimerkki}




\begin{tehtavasivu}

\subsubsection*{Opi perusteet}

\subsubsection*{Hallitse kokonaisuus}

\subsubsection*{Sekalaisia tehtäviä}

TÄHÄN TEHTÄVIÄ SIJOITTAMISTA ODOTTAMAAN

Tehtävissä 1--n esitä lauseke ilman itseisarvomerkkejä.

\begin{tehtava}
	\begin{alakohdat}
		\alakohta{$|\pi-2^2|$}
		\alakohta{$|2x-6|$}
		\alakohta{$x+|6-3x|$}
	\end{alakohdat}
	\begin{vastaus}
		\begin{alakohdat}
			\alakohta{$-(\pi-4)=-\pi+4=4-\pi$}
			\alakohta{$\begin{cases}
					-2x+6, & \jos x<3 \\
					2x-6, & \jos x \geq 3
				\end{cases}$}
			\alakohta{$\begin{cases}
					x+(6-3x), & \jos 6-3x \geq0 \\
					x-(6-3x), & \jos 6-3x <0 
				\end{cases}\\
				=\begin{cases}
					x+6-3x, & \jos -3x \geq-6 \\
					x-6+3x, & \jos -3x <-6 
				\end{cases}\\
				=\begin{cases}
					-2x+6, & \jos x \leq2 \\
					4x-6, & \jos x >2 
				\end{cases}$}
		\end{alakohdat}
	\end{vastaus}
\end{tehtava}

\begin{tehtava}
	\begin{alakohdat}
		\alakohta{$2x-x|2-x|$}
		\alakohta{$|x^2+3|$}
		\alakohta{$|x^2-4|$}
	\end{alakohdat}
	\begin{vastaus}
		\begin{alakohdat}
			\alakohta{$\begin{cases}
					x^2, & \jos x \leq2 \\
					-x^2+4x, & \jos x>2 
				\end{cases}$}
			\alakohta{$x^2+3$}
			\alakohta{$\begin{cases}
					x^2-4, & \jos x \leq -2 \tai x \geq 2 \\
					-x^2+4, & \jos -2<x<2 
				\end{cases}$}
		\end{alakohdat}
	\end{vastaus}
\end{tehtava}

\begin{tehtava}
	\begin{alakohdat}
		\alakohta{$(x-1)|4x^2+4x+1|$}
		\alakohta{$|-3x^2+4x-2|$}
	\end{alakohdat}
	\begin{vastaus}
		\begin{alakohdat}
			\alakohta{$4x^3-3x-1$}
			\alakohta{$3x^2-4x+2$}
		\end{alakohdat}
	\end{vastaus}
\end{tehtava}

\begin{tehtava}
	\begin{alakohdat}
		\alakohta{$3|9-x^2|-2|x^2-9|$, kun $x\leq-3$}
		\alakohta{$\dfrac{|-x^2+4x+5|}{|x^2-5x|}$, kun $x>5$}
	\end{alakohdat}
	\begin{vastaus}
		\begin{alakohdat}
			\alakohta{$x^2-9$ (vinkki: vastalukujen itseisarvot ovat yhtä suuret)}
			%
			\alakohta{$\frac{x+1}{x}$}
		\end{alakohdat}
	\end{vastaus}
\end{tehtava}

\end{tehtavasivu}
	\section{Itseisarvoepäyhtälöt}

\laatikko{
KIRJOITA TÄHÄN LUKUUN

\begin{itemize}
\item miten ratkaistaan epäyhtälöitä tyyliin
\item $|x|<a$,  $|x|>a$
\item $|f(x)|<a$, $|f(x)|<|g(x)|$, $|f(x)|<g(x)$
\end{itemize}

KIITOS!}

\begin{esimerkki}
Ratkaise epäyhtälö $|x|<3$.

\begin{esimratk}
On löydettävä kaikki sellaiset luvut, joiden itseisarvo on pienempi kuin $3$. Esimerkiksi luvut $2$ ja $-1$ käyvät, mutta luvut $4$ ja $-5,5$ eivät käy. Epäyhtälön ratkaisuja ovat täsmälleen ne luvut, joiden etäisyys nollasta on pienempi kuin $3$.

(Tähän kuva.)

Siten ratkaisu on $-3<x<3$.
\end{esimratk}

\begin{esimvast}
$-3<x<3$
\end{esimvast}


\end{esimerkki}

\begin{esimerkki}
Ratkaise epäyhtälö $|x|>5$.

\begin{esimratk}
On löydettävä kaikki sellaiset luvut, joiden itseisarvo on suurempi kuin $5$. Esimerkiksi luvut $2$ ja $-4,5$ eivät ole epäyhtälön ratkaisuja, mutta luvut $5,5$ ja $-6$ ovat. Epäyhtälön ratkaisuja ovat täsmälleen ne luvut, joiden etäisyys nollasta on suurempi kuin $5$.
 
(Tähän kuva.)

Nyt ratkaisu on ilmoitettava kahdessa osassa: $x<-5$ tai $x>5$.
\end{esimratk}

\begin{esimvast}
$x<-5$ tai $x>5$
\end{esimvast}
\end{esimerkki}

\begin{esimerkki}
Ratkaise epäyhtälö $|x+4|<2$.

\begin{esimratk}
Nyt luvun $x+4$ itseisarvo on pienempi kuin kaksi, joten tiedetään, että $-2<x+4<2$. Nyt saadaan ratkaistavaksi epäyhtälöt $-2<x+4$ ja $x+4<2$. Ratkaistaan nämä kaksi yhtälöä erikseen:

\begin{align*}
-2&<x+4 & \ppalkki{+2} \\
0&<x+6 & \ppalkki{-6} \\
-6&<x &
\end{align*}

\begin{align*}
x+4&<2 & \ppalkki{-4} \\
x&<-2 & \ppalkki{-6}
\end{align*}

Kun vastaukset yhdistetään saadaan ratkaisuksi $-6<x<-2$.

(Tähän kuva.)

\end{esimratk}

\begin{esimvast}
$-6<x<-2$.
\end{esimvast}
\end{esimerkki}


\subsection*{Tehtäviä}

\begin{tehtavasivu}

\subsubsection*{Opi perusteet}

\begin{tehtava}
Ratkaise seuraavat epäyhtälöt.
	\begin{alakohdat}
		\alakohta{$|x|<6$}
		\alakohta{$|x|>10$}
		\alakohta{$|x|<1,6$}
	\end{alakohdat}
	\begin{vastaus}
		\begin{alakohdat}
			\alakohta{$-6<x<6$}
			\alakohta{$x<-10$ tai $x>10$}
			\alakohta{$-1,6<x<1,6$}
		\end{alakohdat}
	\end{vastaus}
\end{tehtava}

\begin{tehtava}
Ratkaise seuraavat epäyhtälöt.
	\begin{alakohdat}
		\alakohta{$|x+6|>3$}
		\alakohta{$|x-5|<2$}
	\end{alakohdat}
	\begin{vastaus}
		\begin{alakohdat}
			\alakohta{$x<-9$ tai $x>-3$}
			\alakohta{$3<x<7$}
		\end{alakohdat}
	\end{vastaus}
\end{tehtava}

\subsubsection*{Hallitse kokonaisuus}

\subsubsection*{Sekalaisia tehtäviä}

TÄHÄN TEHTÄVIÄ SIJOITTAMISTA ODOTTAMAAN

\begin{tehtava}
Ratkaise seuraavat epäyhtälöt.
	\begin{alakohdat}
		\alakohta{$|x|\le 6$}
		\alakohta{$|x|>-3$}
	\end{alakohdat}
	\begin{vastaus}
		\begin{alakohdat}
			\alakohta{$-6 \le x \le 6$}
			\alakohta{ei ratkaisua}
		\end{alakohdat}
	\end{vastaus}
\end{tehtava}


\begin{tehtava}
Ratkaise epäyhtälö $|x^2+1| \ge 3$.
	\begin{vastaus}
          $x<\sqrt{-2}$ tai $x>\sqrt{2}$
	\end{vastaus}
\end{tehtava}

\end{tehtavasivu}

	% itseisarvoepäyhtälöt
	\section{Lineaariset yhtälöryhmät} % FIXME: siirrä?

\laatikko{
KIRJOITA TÄHÄN LUKUUN

\begin{itemize}
\item mikä yhtälöryhmä on
\item miten ratkaistaan yhtälöpari (sijoitus, yhteenlaskumenetelmä)
\item että ratkaisuja voi olla yksi, nolla tai äärettömän monta
\item miten useamman tuntemattoman yhtälöryhmä ratkaistaan
\end{itemize}

KIITOS!}

%\subsection*{Yhtälöryhmä}

% \termi{yhtälöryhmä}{Yhtälöryhmällä} tarkoitetaan useaa yhtälöä, joiden täytyy
% päteä samanaikaisesti. Englannin kielessä yhtälöryhmä tunnetaankin nimellä
% \textit{simultaneous equations} eli samanaikaiset yhtälöt.
% \footnote{Termi \textit{system of equations} on vähintäänkin yhtä yleinen.}
% Yhtälöryhmän (mahdollisten) ratkaisujen tulee siis toteuttaa kaikki yhtälöt.
% 
% Tässä luvussa käsitellään yhtälöryhmiä, joissa kaikki yhtälöt ovat ensimmäistä
% astetta. Tällaisia yhtälöryhmiä kutsutaan \termi{lineaarinen yhtälöryhmä}{lineaarisiksi yhtälöryhmiksi}.
% Lisäksi myöhemmin kirjassa käsitellään erikoistapauksia toisen asteen yhtälöryhmistä
% ratkaistaessa kahden ympyrän tai ympyrän ja suoran leikkauspisteitä.

\subsection*{Yhtälöpari}

% Yksinkertaisin mielenkiintoinen yhtälöryhmä on
% \termi{lineaarinen yhtälöpari}{lineaarinen yhtälöpari}.
% Lineaarisessa yhtälöparissa on kaksi ensimmäisen asteen yhtälöä.
% Lineaarinen yhtälöpari voidaan esittää monella tapaa. Tässä
% kirjassa käytämme pääasiallisesti ns. normaalimuotoa.
% 
% \[
% \left\{
% \begin{aligned}
% a_1x+b_1y+c_1 &= 0 \\
% a_2x+b_2y+c_2 &= 0
% \end{aligned}
% \right.
% \]
% 
% $a_1, a_2, b_1, b_2, c_1, c_2 \in \R$
% 
% Lineaarisen yhtälöparin ratkaisu on pari $(x, y) \in \R^2$, joka toteuttaa molemmat yhtälöt.

\begin{esimerkki}
Ratkaise yhtälöpari
\[
\left\{
\begin{aligned}
3x-2y&= 1 \\
-x+5y&= 2.
\end{aligned}
\right.
\]
\begin{esimratk}
On löydettävä kaikki ne luvut $x$ ja $y$ jotka toteuttavat molemmat yhtälöt.

Käytetään \termi{sijoitusmenetelmä}{sijoitusmenetelmää}. Ratkaistaan muuttuja $x$ alemmasta yhtälöstä ja sijoitetaan se ylempään yhtälöön. Alemmasta yhtälöstä $-x+5y= 2$ saadaan $x=5y-2$. Sijoitetaan tämä ylempään yhtälöön $3x-2y=1$:
\begin{align*}
3x-2y&=1 && \ppalkki \text{Sijoitetaan $x=5y-2$.} \\
3(5y-2)-2y&=1 && \\
15y-6-2y&=1 && \\
13y&=7 && \\
y&=\frac{13}{7} && \\
\end{align*}
Sijoitetaan $y=13/7$ alempaan yhtälöön, jotta voidaan ratkaista $x$:
\begin{align*}
-x+5y&= 2 && \ppalkki \text{Sijoitetaan $y=\frac{13}{7}$.} \\
-x+\frac{65}{7}&= 2 && \\
-x&= -\frac{51}{7}&& \\
x&= \frac{51}{7}&&
\end{align*}
\end{esimratk}
\begin{esimvast}
Yhtälön ratkaisu on $x= 51/7$, $y=13/7$.
\end{esimvast}
\end{esimerkki}

\begin{esimerkki}
Määritä suorien $-2x-y= 4$ ja $3x-2y=1$ leikkauspiste.
\begin{esimratk}
Saadaan ratkaistavaksi yhtälöpari
\[
\left\{
\begin{aligned}
-2x-y&= 4 \\
3x-2y&= 1.
\end{aligned}
\right.
\]
Käytetään \termi{eliminointimenetelmä}{yhteenlaskumenetelmää}, jolla voidaan eliminoida toinen muuttujista.
\begin{align*}
&\left\{
\begin{aligned}
2x-y&= 4 && \ppalkki \cdot 3\\
3x-2y&= 1. && \ppalkki \cdot (-2)
\end{aligned}
\right. \\
&\left\{
\begin{aligned}
6x-3y&= 12 && \ppalkki \text{Lasketaan yhtälöt yhteen.}\\
-6x+4y&= -2.&&
\end{aligned}
\right.\\
& y&= 10 \\
\end{align*}
Sijoitetaan $y=10$ jompaan kumpaan alkuperäisistä yhtälöistä, jotta saadaan ratkaistua $x$. Käytetään vaikkapa yhtälöä $3x-2y=1$:
\begin{align*}
3x-2y&=1 && \ppalkki \text{Sijoitetaan $x=10$.} \\
3x-20&=1 && \\
3x&=21 && \\
x&=7.
\end{align*}
\end{esimratk}
\begin{esimvast}
Yhtälön ratkaisu on $x=7$, $y=10$.
\end{esimvast}
\end{esimerkki}

Aiemmin on todettu, että normaalimuotoinen ensimmäisen asteen yhtälö voidaan tulkita suorana
$(x, y)$-tasossa. Näin ollen lineaariselle yhtälöparille on geometrinen tulkinta: sen
ratkaisut ovat ne tason pisteet, joissa yhtälöitä vastaavat
suorat leikkaavat. Näitä voi olla
$0$ (suorat ovat yhdensuuntaiset, mutta eivät sama suora),
$1$ (suorat eivät ole yhdensuuntaiset) tai
äärettömän monta (suorat ovat sama suora).

Lineaarisia yhtälöpareja ratkotaan pääasiallisesti kahdella menetelmällä.
Nämä menetelmät ovat \termi{sijoitusmenetelmä}{sijoitusmenetelmä} ja
\termi{yhteenlaskumenetelmä}{yhteenlaskumenetelmä}.


\subsection*{Lineaarinen yhtälöryhmä}

Ratkaistavia yhtälöitä voi olla useampia kuin kaksi, ja tällöin puhutaan \termi{yhtälöryhmä}{yhtälöryhmästä}. Yhtälöpari on siis yhtälöryhmän erikoistapaus.

Jos yhtälöt ovat lisäksi ensimmäistä astetta, on kyseessä \termi{lineaarinen yhtälöryhmä}{lineaarinen yhtälöryhmä}.
Esimerkiksi
\[
\left\{
\begin{aligned}
3x-2y&= -5 \\
5x+6y&= 1 \\
-x+5y&= 0.
\end{aligned}
\right.
\]
on lineaarinen yhtälöryhmä.

% Tässä tarkastellaan lähinnä kolmen yhtälön lineaarisia yhtälöryhmiä. Neljän yhtälön
% lineaarisista yhtälöryhmistä esitetään joitakin helppoja esimerkkejä. Yleisesti ottaen yhtälöryhmiä
% ei ratkaista tällä kurssilla esitetyin keinoin, vaan likimääräisesti tietokoneella käyttäen numeerista 
% matriisilaskentaa, joka ei kuulu lukion oppimäärään.

\subsection*{Lineaarisen yhtälöryhmän ratkaisumenetelmiä}

% tähän menetelmistä

\begin{tehtavasivu}

\subsubsection*{Opi perusteet}

\subsubsection*{Hallitse kokonaisuus}

\subsubsection*{Sekalaisia tehtäviä}

TÄHÄN TEHTÄVIÄ SIJOITTAMISTA ODOTTAMAAN

\begin{tehtava}
    Ratkaise yhtälöpari.
    \begin{align*}
        x+y+1 &= 0 \\
        x+2y+1 &=0
    \end{align*}
    \begin{vastaus}
        $x = -1, \, y = 0$
    \end{vastaus}
\end{tehtava}

\begin{tehtava}
    Ratkaise yhtälöpari.
    \begin{align*}
        2x+5y+1 &= 0 \\
        2x+2y+7 &=0
    \end{align*}
    \begin{vastaus}
        $x = -\frac{11}{2}, \, y = 2$
    \end{vastaus}
\end{tehtava}

\begin{tehtava}
    Ratkaise yhtälöpari. $t \in \R$ on vapaa parametri, joka saa sisältyä vastaukseen.
    \begin{align*}
        x+2y-t-1 &= 0 \\
        x+y+t^2 &=0
    \end{align*}
    \begin{vastaus}
        $x = -2t^2-t-1, \, y = t^2+t+1$
    \end{vastaus}
\end{tehtava}

\begin{tehtava}
    Ratkaise yhtälöryhmä.
	$$\left\{    
    \begin{array}{rcl}
        x+2y+1 &=&0 \\
        x+2z+3 &=&0 \\
        y+2z+5 &=&0
    \end{array}
    \right.$$
    \begin{vastaus}
        $x = 1, \, y = -1, \, z = -2$
    \end{vastaus}
\end{tehtava}

\begin{tehtava}
    Ratkaise yhtälöryhmä.
    \begin{align*}
        x+y+z+8 &= 0 \\
        x+y+6 &=0 \\
        x+z-70 &=0
    \end{align*}
    \begin{vastaus}
        $x = 72, \, y = -78, \, z = -2$
    \end{vastaus}
\end{tehtava}

\begin{tehtava}
    Ratkaise yhtälöryhmä.
    \begin{align*}
        x+y+2z+12 &= 0 \\
        2x+2y+3z+1 &=0 \\
        3x-4 &=0
    \end{align*}
    \begin{vastaus}
        $x = \frac{4}{3}, \, y = \frac{98}{3}, \, z = -23$
    \end{vastaus}
\end{tehtava}

\begin{tehtava}
    Ratkaise yhtälöryhmä.
    \begin{align*}
        2x+3y+5z+8 &= 0 \\
        3x+5y+8z &=0 \\
        x+y-1 &=0
    \end{align*}
    \begin{vastaus}
        $x = -\frac{63}{2}, \, y = \frac{65}{2}, \, z = -\frac{17}{2}$
    \end{vastaus}
\end{tehtava}

\end{tehtavasivu}

	% sijoitusmenetelmä
	% yhtälöiden laskeminen yhteen
	% ratkaisujen määrä
	\section{Koordinaatisto ja yhtälön kuvaaja}

\laatikko{
KIRJOITA TÄHÄN LUKUUN

\begin{itemize}
\item ihan lyhyt koordinaatistokertaus
\item kahden pisteen välinen etäisyys (pysty-tai vaakasuoraan helpolla, Pythagoraan lauseella yleensä)
\item esimerkkejä käyrän yhtälöistä, esim. suora, paraabeli, kartesiuksen lehti
\end{itemize}

KIITOS!}

Analyyttisen geometrian perusajatus on käsitellä geometrisia kuvioita koordinaatistossa.
Koordinaatistossa kuvion jokainen piste voidaan ilmoittaa sen koordinaattien avulla.
Esimerkiksi oheisessa kuvassa on kolmio $ABC$, jonka nurkkapisteiden koordinaatit ovat
\[
A(1, 2), \quad B(-1, -1) \quad \text{ja} \quad C(2, -2).
\]

\begin{kuva}
    kuvaaja.pohja(-2, 3, -3, 3, korkeus = 4, nimiX = "$x$", nimiY = "$y$", ruudukko = True)
    geom.jana((1, 2), (-1, -1))
    geom.jana((-1, -1), (2, -2))
    geom.jana((2, -2), (1, 2))
    kuvaaja.piste((1, 2), "$A$", 45)
    kuvaaja.piste((-1, -1), "$B$", 180)
    kuvaaja.piste((2, -2), "$C$", -45)
\end{kuva}

Koordinaattiakselit leikkaavat toisensa kohtisuoraan pisteessä, jota nimitetään \termi{origo}{origoksi}.
Tuota pistettä voi ajatella koordinaatiston keskuksena.
Vaaka-akselia on yleensä tapana nimittää $x$-akseliksi ja pystyakselia $y$-akseliksi.
Näiden mukaan koko koordinaatistoa kutsutaan toisinaan $xy$-koordinaatistoksi.

Kunkin pisteen koordinaatit määräytyvät siitä, missä kohdassa se on $x$- ja $y$-akselien asteikkoihin verrattuna.
Aivan kuten lukusuoralla kutakin pistettä vastaa tietty reaaliluku $x$ ja päinvastoin, koordinaatistossa kutakin pistettä vastaa yksi yhteen tietty lukupari $(x, y)$.

\subsection{Pisteiden välinen etäisyys}

Geometriassa tärkeää on päästä mittaamaan pituuksia.
Tätä varten on selvitettävä, miten voidaan määrittää kahden koordinaatiston pisteen välinen etäisyys.
Aivan kuten tavallisessa geometriassa, emme voi aina turvautua mittaamiseen, sillä siten ei saada täsmällisiä tuloksia.
Sen sijaan pyritään selvittämään pisteiden väliset etäisyydet niiden koordinaattien perusteella.

\begin{kuva}
    kuvaaja.pohja(-4, 3, 0, 6, korkeus = 4, nimiX = "$x$", nimiY = "$y$", ruudukko = True)
    piste((1, 2), "$A$")
    piste((1, 5), "$B$")
    piste((-3, 2), "$C$")
\end{kuva}

Helpointa etäisyyden määrittäminen on silloin, kun pisteet ovat samalla vaaka- tai pystysuoralla.
Toisin sanoen niillä on sama $x$- tai $y$-koordinaatti.
Tällöin niiden etäisyys saadaan yksinkertaisesti laskemalla toisistaan poikkeavien koordinaattien erotus.

Esimerkiksi yllä olevassa kuvassa pisteiden $A(1, 2)$ ja $B(1, 5)$ välinen etäisyys on $5-2=3$.
Toisaalta pisteiden $A(1, 2)$ ja $C(-3, 2)$ välinen etäisyys on $1-(-3)=4$.

Etäisyyden tulee olla aina positiivinen.
Jos ei ole varma pisteiden järjestyksestä, voi käyttää itseisarvoja.
Esimerkiksi edellä pisteiden $A$ ja $C$ etäisyys voidaan laskea myös järjestyksessä $|-3-1|=|-4|=4$.

% \begin{esimerkki}
% jotkin helpot etäisyydet
% \end{esimerkki}

\begin{kuva}
    kuvaaja.pohja(-2, 3, -1, 5, korkeus = 4, nimiX = "$x$", nimiY = "$y$", ruudukko = True)
    piste((1, 2), "$A$")
    piste((4, 3), "$B$")
    geom.jana((1, 2), (1, 3))
    geom.jana((1, 3), (4, 3))
    geom.jana((4, 3), (1, 2))
\end{kuva}

Kun kahdella pisteellä on sekä eri $x$-koordinaatit että eri $y$-koordinaatit, etäisyys on määritettävä toisella tapaa.
Nyt voidaan turvautua Pythagoraan lauseeseen ja siihen, että koordinaatisto on suorakulmainen.

Edellä olevassa kuvassa pisteiden $A(1, 2)$ ja $B(4, 3)$ etäisyys saadaan piirtämällä kuvan mukainen suorakulmainen kolmio $ABC$.
Kateetin $AC$ pituus on pisteiden $A$ ja $B$ $x$-koordinaattien erotus eli $4-1=3$.
Kateetin $BC$ pituus on puolestaan $A$ ja $B$ $y$-koordinaattien erotus eli $3-2=1$.
Hypotenuusan $AB$ pituus saadaan Pythagoraan lauseesta:
\[
|AB|=\sqrt{3^2+1^2}=\sqrt{9+1}=\sqrt{10}.
\]
Pisteiden $A$ ja $B$ välinen etäisyys on siis $\sqrt{10}$.
Tätä ei olisi voinut selvittää mittaamalla.

Yleisessä tapauksessa kahden pisteen välinen etäisyys saadaan seuraavasta kaavasta.
\laatikko{
\textbf{Pisteiden $(x_1, y_1)$ ja $(x_2, y_2)$ välinen etäisyys.}
\[
\sqrt{(x_1-x_2)^2+(y_1-y_2)^2}
\]
}

\begin{esimerkki}
Mitkä ovat luvun alussa olleen kolmion $ABC$ sivujen pituudet?

\begin{esimratk}
Pisteiden koordinaatit olivat $A(1, 2)$, $B(-1, -1)$ ja $C=(2, -2)$.
Kolmion sivujen pituudet ovat sen kärkipisteiden väliset etäisyydet.
Yllä olevaa kaavaa soveltamalla saadaan sivun $AB$ pituudeksi
\[
\sqrt{\bigl(1-(-1)\bigr)^2+\bigl(2-(-1)\bigr)^2}=\sqrt{2^2+3^2}=\sqrt{4+9}=\sqrt{13}.
\]
Sivun $AC$ pituudeksi tulee
\[
\sqrt{(1-2)^2+\bigl(2-(-2)\bigr)^2}=\sqrt{(-1)^2+4^2}=\sqrt{1+16}=\sqrt{17}.
\]
Sivun $BC$ pituudeksi tulee
\[
\sqrt{\bigl((-1)-2\bigr)^2+\bigl((-1)-(-2)\bigr)^2}=\sqrt{(-3)^2+1^2}=\sqrt{9+1}=\sqrt{10}.
\]
\end{esimratk}

\begin{esimvast}
Sivujen pituudet ovat $|AB|=\sqrt{13}$, $|AC|=\sqrt{17}$ ja $|BC|=\sqrt{10}$.
\end{esimvast}
\end{esimerkki}

\subsection{Käyrät ja niiden yhtälöt}

Koordinaatistoon voidaan pisteiden ja janojen lisäksi piirtää myös suoria ja erilaisia käyriä.
Esimerkiksi polynomien kuvaajia piirrettiin ja tutkittiin pitkän matematiikan kurssissa 2 ja suoriin on tutustuttu jo aiemmin.
Yhteistä kaikille näille on se, että koordinaatistoon piirrettyä käyrää (jollaiseksi myös suora voidaan laskea) vastaa jokin yhtälö, jossa esiintyvät tuntemattomat $x$ ja $y$.

Alla olevan kuvan suoraa vastaa yhtälö $y=2x-1$.
Yhtälö tulkitaan siten, että jokaisen suoralla olevan pisteen $(x, y)$ koordinaatit toteuttavat yhtälön.
Esimerkiksi suoran piste $(2, 3)$ toteuttaa yhtälön, sillä $3=2\cdot 2-1$.
Toisaalta piste $(5, 2)$ ei ole suoralla, sillä $2\cdot 5-1=9\neq 2$.

[TÄHÄN KUVA SUORASTA, JOHON ON MERKITTY PISTE (1,1)]

Kääntäen voidaan sanoa, että jos piste toteuttaa yhtälön $y=2x-1$, se on suoralla.
Yhtälö siis täysin määrittää suoran pisteet.
Etsittäessä pisteitä, jotka muodostavat suoran, riittää tutkia sen yhtälöä.

Monilla tutuilla käyrillä on omat yhtälönsä.
Alle on piirretty paraabeli, jonka yhtälö on $y=x^2-2x+1$.
Aivan kuten suoran tapauksessa, paraabeli koostuu täsmälleen niistä pisteistä, jotka toteuttavat tämän yhtälön.
Toisinaan sanotaan, että paraabeli on \emph{niiden pisteiden joukko, jotka toteuttavat yhtälön $y=x^2-2x+1$}.

[TÄHÄN PARAABELI]

Tällä kurssilla opitaan käsittelemään erityisesti suoria ja ympyröitä niiden yhtälöiden avulla.
Ympyrän yhtälö muodostetaan vaatimuksesta, että jokainen ympyrän piste on yhtä kaukana ympyrän keskipisteestä.
Tällöin tulee käyttöön edellä opittu kahden pisteen etäisyyden kaava.
Esimerkiksi alla olevassa kuvassa on piirretty ympyrä, jonka yhtälö on $(x-1)^2+(y-2)^2=4$.

[TÄHÄN YMPYRÄ]

Yllä olevasta ympyrän yhtälöstä nähdään, että käyrien yhtälöt eivät ole välttämättä muotoa $y=f(x)$.
Toinen esimerkki on niin sanottu Cartesiuksen lehti, jonka yhtälö on $x^3+y^3-3xy=0$.
Kartesiuksen lehti on piirretty alla olevaan kuvaan.

[TÄHÄN CARTESIUKSEN LEHTI]

Esimerkiksi piste $(3/2, 3/2)$ on käyrällä, sillä
\[
\left(\frac{3}{2}\right)^3+\left(\frac{3}{2}\right)^3-3\cdot\frac{3}{2}\cdot\frac{3}{2}
=\frac{27}{8}+\frac{27}{8}-\frac{27}{4}=\frac{27+27-54}{8}=0.
\]

\begin{tehtavasivu}

\subsubsection*{Opi perusteet}

\subsubsection*{Hallitse kokonaisuus}

\subsubsection*{Sekalaisia tehtäviä}

TÄHÄN TEHTÄVIÄ SIJOITTAMISTA ODOTTAMAAN

\end{tehtavasivu}
	% yleistä käyristä, esim. Kartesiuksen lehdestä jotain
	% kahden pisteen välinen etäisyys (Pythagoraan lauseella)

\chapter{Suorat}
	\section{Suoran yhtälö}

	% ratkaistu muoto y = kx + b, kulmakerroin ja vakiotermi
	% nollakohdat ja leikkauspisteet
	% vaaka- ja pystysuorat
	\section{Suoran yhtälön muut muodot}

\laatikko{
KIRJOITA TÄHÄN LUKUUN

\begin{itemize}
\item suoran yhtälö normaalimuodossa $ax+by+c=0$
\item suoran yhtälö muodossa $y-y_0=k(x-x_0)$, suoran yhtälön muodostaminen pisteiden avulla
\end{itemize}

KIITOS!}

%\subsection*{Valeorigomuoto} % tämä ei ole vakiintunut termi!!!

%Toisinaan suoran yhtälöä on helpompaa tarkastella muodossa $y-y_0=k(x-x_0)$.
%Tämän voidaan ajatella olevan origon kautta kulkeva suora, jos origo olisi
%pisteessä $(x_0, y_0)$. Valeorigomuoto on kätevin silloin, kun tiedämme suoran
%kulmakertoimen ja yhden pisteen, jonka kautta suora kulkee.

%\begin{esimerkki}
%    Esimerkkejä valeorigomuodon käytöstä:
%    \begin{enumerate}[a)]
%        \item Suoran kulmakerroin on $5$ ja suora kulkee pisteen $(3, 0)$ kautta.
%        \[y-y_0 = k(x-x_0) \ekvi y-0 = 5(x-3) \ekvi y = 5x-15\]
%        \item Suoran kulmakerroin on $4$ ja suora kulkee pisteen $(5, 7)$ kautta.
%        \[y-y_0 = k(x-x_0) \ekvi y-7 = 4(x-5) \ekvi y-7 = 4x-20 \ekvi y = 4x-13\]
%    \end{enumerate}
%\end{esimerkki}

\subsection*{Suoran yhtälön määrittäminen}

\begin{esimerkki}
Suoran kulmakerroin on $4$ ja se kulkee pisteen $(-3, 2)$ kautta. Määritä suoran yhtälö.

\begin{esimratk}
Olkoon piste $(x, y)$ suoralla. Nyt suoran kulmakerroin on 
\[
\frac{y-2}{x-(-3)}.
\]
Toisaalta tiedetään kulmakertoimen olevan 4. Tästä saadaan yhtälö
\[
\frac{y-2}{x+3}=4.
\]
Ratkaistaan yhtälö:
\begin{align*}
\frac{y-2}{x+3}&=4 \\
y-2&=4(x+3) \\
y-2&=4x+12 \\
y&=4x+14. \\
\end{align*}
\end{esimratk}

\begin{esimvast}
Suoran yhtälö on $y=4x+14$.
\end{esimvast}
\end{esimerkki}

Samanlainen kaava voidaan johtaa yleisemmin. Oletetaan, että suoran kulmakerroin on $k$ ja se kulkee pisteen $(x_0, y_0)$ kautta. Määritetään suoran yhtälö.
Samaan tapaan kuin edellisessä esimerkissä saadaan yhtälö
\[
\frac{y-y_0}{x-x_0}=k.
\]
Ratkaistaan se:
Ratkaistaan yhtälö:
\begin{align*}
\frac{y-y_0}{x-x_0}&=k && \ppalkki{\cdot (x-x_0)}\\
y-y_0&=k(x-x_0) && \ppalkki{+y_0} \\
y-y_0&=k(x-x_0). &&
\end{align*}

\laatikko{
Jos suora kulkee pisteen $(x_0, y_0)$ ja sen kulmakerroin on $k$, suoran yhtälö on
\[y-y_0=k(x-x_0).\]
}

\begin{esimerkki}
Suora kulkee pisteiden $(3, 4)$ ja $(-1, 5)$ kautta. Määritä suoran yhtälö.

\begin{esimratk}
Lasketaan ensin suoran kulmakerroin $k$. Se on
\[
k=\frac{-5+4}{-1-3}=\frac{1}{4}.
\]
Käytetään sitten edellä johdettua suoran yhtälön kaavaa $y-y_0=k(x-x_0)$, missä $(x_0, y_0)$ on jokin suoralla oleva piste. Tässä tapauksessa pisteeksi voidaan valita kumpi tahansa pisteistä $(3, 4)$ ja $(-1, 5)$.
Valitaan vaikkapa $(x_0, y_0)=(3, 4)$. Nyt suoran yhtälö on $y-4=\frac{1}{4}(x-3)$. Se saadaan edelleen muotoon $y=\frac{1}{4}(x-3)+4$.
\end{esimratk}
\begin{esimvast}
Suoran yhtälö on $y=\frac{1}{4}(x-3)+4$.
\end{esimvast}
\end{esimerkki}

\subsection*{Normaalimuoto}

Suoran yhtälön voi kirjoittaa monessa eri muodossa. Esimerkiksi $y=3x-4$, $y-4=3x$ ja $3x-y-4=0$ ovat kaikki saman suoran yhtälöitä. Viimeistä niistä kutsutaan 
suoran yhtälön normaalimuodoksi.

\laatikko{Suoran yhtälön normaalimuoto on
\[ ax+by+c=0, \]
missä joko $a \neq 0$ tai $b \neq 0$.}

\begin{esimerkki}
Kirjoitetaan suoran $y=-5x+2$ yhtälö normaalimuodossa.
\begin{esimratk}
\begin{align*}
y&=-5x+2 && \ppalkki{+5x} \\
5x+y&=2 && \ppalkki{-2} \\
5x+y-2&=0 &&
\end{align*}
\end{esimratk}
\begin{esimvast}
Suoran yhtälön normaalimuoto on $5x+y-2=0$.
\end{esimvast}
\end{esimerkki}


Mikä tahansa suora voidaan kirjoittaa normaalimuodossa. Esimerkiksi pystysuoraa $x=-3$ ei voi kirjoittaa muodossa $y=kx+b$, sillä suoralla ei ole kulmakerrointa. Suora voidaan kuitenkin kirjoittaa normaalimuodossa. Sen normaalimuoto on $x+3=0$.

\begin{tehtavasivu}

\subsubsection*{Opi perusteet}

\begin{tehtava}
Mikä on suoran yhtälö normaalimuodossa?
\begin{enumerate}[a)]
\item $y=-15x+2$
\item $2y=11x+7$
\item $2y+5x-8=13y-6x-8$
\end{enumerate}
\begin{vastaus}
a)$15x+y-2=0$ b) $11x-2y+7=0$ c) $-11x+11y=0$
\end{vastaus}
\end{tehtava}

\begin{tehtava}
Suoran kulmakerroin on $\frac{1}{2}$ ja suora kulkee pisteen
\begin{enumerate}[a)]
\item $(-12, 4)$
\item $(3, 9)$. Mikä on suoran yhtälö?
\end{enumerate}
\begin{vastaus}
a)$y=\frac{1}{2}x+10$ b) $y=\frac{1}{2}x+\frac{15}{2}$
\end{vastaus}
\end{tehtava}

\begin{tehtava}
Mikä on pisteiden
\begin{enumerate}[a)]
\item $(1, -2)$ ja $(3, 1)$
\item $(0, 0)$ ja $(-4, 4)$ kautta kulkevan suoran yhtälö?
\end{enumerate}
\begin{vastaus}
a)$y=\frac{3}{2}x-\frac{7}{2}$ b) $y=-x$
\end{vastaus}
\end{tehtava}

\subsubsection*{Hallitse kokonaisuus}

\begin{tehtava}
Tutki ovatko pisteet  
\begin{enumerate}[a)]
\item $(1, -5)$, $(4, -23)$ja $(4, -239)$
\item $(7, 3)$, $(-2, 10)$ ja $(-3, 90)$ samalla suoralla?
\end{enumerate}
\begin{vastaus}
a) kyllä b) ei
\end{vastaus}
\end{tehtava}

\begin{tehtava}
Määritä luku $t$ niin, että pisteet $(-t+3, -4)$, $(6, t-5)$ ja $(5, -4)$ ovat samalla suoralla.
\begin{vastaus}
$t=-2$ tai $t=1$
\end{vastaus}
\end{tehtava}

\subsubsection*{Sekalaisia tehtäviä}

TÄHÄN TEHTÄVIÄ SIJOITTAMISTA ODOTTAMAAN

\begin {tehtava}
Suora kulkee pisteiden $(3, 4)$ ja $(\sqrt{3}, 1)$ kautta. Määritä suoran kulmakerroin.
\begin {vastaus}
$\frac{\sqrt{3}-1}{\sqrt{3}}$
\end {vastaus}
\end {tehtava}

\end{tehtavasivu}

	% esitys y-y_0=k(x-x_0)
	% esitys ax + by + c = 0 (normaalimuoto)
	\section{Suorien keskinäinen asema}

\laatikko{
KIRJOITA TÄHÄN LUKUUN

\begin{itemize}
\item sama kulmakerroin --> yhdensuuntaiset tai sama suora
\item eri kulmakerroin --> yksi leikkauspiste, kytkentä yhtälöpareihin
\item suoran ja normaalin kulmakertoimet, $k_1k_2=-1$.
\end{itemize}

KIITOS!}
	% suorien keskinäinen asema, yhdensuuntaiset suorat
	% suoralle ja sen normaalille k_1 * k_2 = -1
	\section{Pisteen etäisyys suorasta}

\laatikko{
KIRJOITA TÄHÄN LUKUUN

\begin{itemize}
\item pisteen etäisyyden suorasta laskeminen yhdenmuotoisilla
kolmioilla
\item pisteen etäisyys suorasta -kaava: $d=\frac{|ax_0+by_0+c|}{\sqrt{a^2+b^2}}$
\item sovelluksia
\item kaavan todistuksen voi laittaa tähän osioon tai liitteeksi,
käytetään yhdenmuotoisia kolmioita
\end{itemize}

KIITOS!}

Pisteen etäisyydellä suorasta tarkoitetaan pisteen ja mielivaltaisen suoran pisteen pienintä mahdollista etäisyyttä.
Jos tunnetaan jokin vaaka- tai pystysuora suora ja jokin koordinaatiston piste, kyseisen pisteen etäisyys annetusta suorasta on helppo määrittää.

[TÄHÄN SELITYS JA KUVA]

Jos suora on kalteva, etäisyyden määrittäminen ei ole näin suoraviivaisesti.
Seuraavaksi tutustutaan kahteen tapaan tämän pulman ratkaisemiseksi.

\subsection*{Pisteen etäisyys suorasta yhdenmuotoisten kolmioiden avulla}

Tarkastellaan esimerkkinä suoraa $l$, jonka normaalimuotoinen yhtälö on $3x-4y=12$. [TÄMÄ EI MUKA OLE NORMAALIMUOTO. KAI KORJATTAVA]
Selvitetään pisteen $P=(8, 5)$ etäisyys suorasta $l$.

[TÄHÄN KUVA]

Kuvaan on merkitty suorakulmaiset kolmiot $OAB$ ja $PQR$.
Etäisyys, jonka haluamme selvittää, on kolmion $PQR$ sivun $r$ pituus.
Tehtävä ratkeaa, kun huomataan, että kolmiot $OAB$ ja $PQR$ ovat yhdenmuotoisia.
Tämä johtuu siitä, että molemmat ovat suorakulmaisia ja lisäksi kulmat $OAB$ ja $PRQ$ ovat samankokoiset (ks. kuva alla).

[TÄHÄN KUVA]

Kolmion $OAB$ sivut selviävät, kun ratkaistaan, missä pisteissä suora leikkaa $x$- ja $y$-akselit.
Asettamalla suoran yhtälössä $x=0$ suoran yhtälössä
\[
3x-4\cdot 0=12, \quad \text{josta} \quad x=\frac{12}{3}=4.
\]
Pisteen $A$ koordinaatit ovat siis $(4, 0)$. Toisaalta kun $x=0$, saadaan
\[
3\cdot 0-4\cdot y=12, \quad \text{josta} \quad y=-\frac{12}{4}=3.
\]
Pisteen $B$ koordinaatit ovat siis $(0, 3)$. Nyt tunnetaan sivut $a=4$ ja $b=3$, ja lisäksi Pythagoraan lauseen perusteella
\[
c=\sqrt{a^2+b^2}=\sqrt{4^2+3^2}=\sqrt{25}=5.
\]

Koska kolmiot $OAB$ ja $PQR$ ovat yhdenmuotoiset, saadaan verranto
\[
\frac{r}{q}=\frac{a}{c}.
\]
Tunnemme jo sivut $a$ ja $c$, joten enää on selvitettävä sivu $q$. Tämä on sama kuin pisteiden $P$ ja $R$ välinen etäisyys.

Pisteen $R$ $x$-koordinaatti on sama kuin pisteen $P$, eli 8. Koska $R$ on suoralla $l$, sen $y$-koordinaatti saadaan suoran yhtälöstä:
\begin{align*}
3\cdot 8-4y & =12 \\
-4y & =12-3\cdot 8 \\
-4y & =-12 \\
y & =3. \\
\end{align*}
Nyt siis $R=(6, 3)$. Pisteiden $P$ ja $R$ välinen etäisyys on siis $5-3=2$, ja tämä on sivun $q$ pituus.

Kun verrantoon $\dfrac{r}{q}=\dfrac{a}{c}$ sijoitetaan tunnetut sivujen pituudet, saadaan
\begin{align*}
\frac{r}{2} & =\frac{4}{5} \quad \ppalkki \cdot 2 \\[3pt]
r & =\frac{8}{5}.
\end{align*}
Siispä pisteen $P$ etäisyys suorasta $l$ on $\dfrac{8}{5}$.

\subsection*{Pisteen etäisyys suorasta kaavan avulla}

Edellä esitetystä tavasta laskea pisteen etäisyys suorasta voidaan johtaa myös kaava.
Jos suoran yhtälö on annettu normaalimuodossa $Ax+By+C=0$ ja pisteen koordinaatit ovat $(x_0, y_0)$, etäisyys $d$ saadaan seuraavasta kaavasta.
\laatikko[pisteen etäisyys suorasta]{
\[
d=\frac{|Ax_0+By_0+C|}{\sqrt{A^2+B^2}}
\]
}
Kaavan johtaminen esitetään liitteessä. (VAI TÄSSÄ?)

\begin{esimerkki} Lasketaan aiemman esimerkin pisteen $P=(8, 5)$ etäisyys suorasta $l$, jonka normaalimuotoinen yhtälö on $3x-4y-12=0$.
\begin{esimratk}
Käytetään kaavaa, jolloin $A=3$, $B=-4$ ja $C=12$, sekä $x_0=8$ ja $y_0=5$. Kaavan mukaan etäisyys on
\[
d=\frac{|Ax_0+By_0+C|}{\sqrt{A^2+B^2}}
=\frac{|3\cdot 8-4\cdot 5-12|}{\sqrt{3^2+(-4)^2}}
=\frac{|24-20-12|}{\sqrt{9+16}}=\frac{|-8|}{\sqrt{25}}
=\frac{8}{5}.
\]
\end{esimratk}
\begin{esimvast}
Etäisyys on $\dfrac{8}{5}$.
\end{esimvast}
\end{esimerkki}

\begin{esimerkki} Etsitään ne pisteet, joiden $x$-koordinaatti on 6 ja joiden etäisyys suorasta $-6x+8y+3=0$.
\begin{esimratk}
Piste, jonka $x$-koordinaatti on 6, on muotoa $(6, y)$. Sijoitetaan tämä etäisyyden kaavaan ja sievennetään.
Nyt $A=-6$, $B=8$, $C=3$, $x_0=6$ ja $y_0=y$.
\[
d=\frac{|-6\cdot 6+8y+3|}{\sqrt{(-6)^2+8^2}}
=\frac{|-36+8y+3|}{\sqrt{36+64}}
=\frac{|8y-33|}{\sqrt{100}}
=\frac{|8y-33|}{10}.
\]
Tehtävänannon mukaan etäisyyden pitäisi olla $d=6$. Tästä saadaan yhtälö
\[
\frac{|8y-33|}{10}=6 \quad \text{eli} \quad |8y-33|=60.
\]
Tämä itseisarvoyhtälö ratkeaa jakautumalla kahteen tapaukseen:
\begin{align*}
8y-33 & =60 & &\text{tai} & 8y-33 & =-60 \\
8y & =99 & & & 8y & =-27 \\
y & =\frac{99}{8} & & & y & =-\frac{27}{8}.
\end{align*}
\end{esimratk}
\begin{esimvast}
Pisteet ovat $\bigl(6, \frac{99}{8}\bigr)$ ja $\bigl(6, -\frac{27}{8}\bigr)$.
\end{esimvast}
\end{esimerkki}



\subsection*{Pisteen etäisyys suorasta, kaavan todistus}

Lasketaan pisteen $P = (x_0, y_0)$ etäisyys suorasta $l$: $ax+by+c=0$.

Olkoon $Q$ suoralla $l$ siten, että $l$ ja $PQ$ ovat kohtisuorassa. Huomataan, että jos piste $R$ on suoralla $l$, Pythagoraan lauseen mukaan
\[
PR^2 = PQ^2+QR^2 \geq PQ^2,
\]
jolloin myös $PR \geq PQ$. Siis $PQ$ on määritelmän nojalla pisteen $P$ etäisyys suorasta $l$.

Suorat $PQ$ ja $l$ ovat kohtisuorassa. Jos $l$ ei ole $x$-, eikä $y$-akselin suuntainen, eli $a,b \neq 0$ sen kulmakerroin on $-\frac{a}{b}$. Koska suoran ja normaalin kulmakerrointen tulo on $-1$, suoran $PQ$ kulmakerroin on
\[
-\frac{1}{\frac{-a}{b}} = \frac{b}{a}.
\]
Se kulkee lisäksi pisteen $(x_0,y_0)$ kautta, joten sen yhtälö on
\[
y-y_0 = \frac{b}{a}(x-x0)
\]
tai normaalimuodossa
\[
bx-ay+ay_0-bx_0 = 0.
\]
Toisaalta, jos $a = 0$, suora $PQ$ on muotoa $x+C$, jollain reaaliluvulla $C$; koska se lisäksi kulkee pisteen $P$ kautta, sen on oltava edellistä muotoa. Sama päättely voidaan toistaa kun $b = 0$, jolloin nähdään, että kaava normaalille pätee myös jos $a = 0$ tai $b = 0$.

Koska $Q$ kuuluu suoralle $l$ ja sen normaalille, sen koordinaattien on toteutettava yhtälöpari
\[
\left\{    
    \begin{array}{rcl}
        ax_q + by_q + c &=&0 \\
        bx_q-ay_q+ay_0-bx_0 &=& 0 \\
    \end{array}
    \right.
\]
Yhtälöpari voidaan ratkaista yhtäänlaskumenetelmällä kertomalla ylempi yhtälö $a$:lla ja alempi $b$:llä.
\[
\left\{    
    \begin{array}{rcl}
        a^2x_q + aby_q + ac &=&0 \\
        b^2x_q-aby_q+aby_0-b^2x_0 &=& 0 \\
    \end{array}
    \right.
\]
ja laskemalle yhtälöt yhtälöt yhteen
\begin{align*}
 a^2x_q + aby_q + ac + b^2x_q-aby_q+aby_0-b^2x_0 &= 0 \\
 (a^2+b^2)x_q &= -ac-aby_0+b^2x0 \\
 x_q = \frac{-ac-aby_0+b^2x0}{(a^2+b^2)}.
\end{align*}
Vastaavasti
\[
y_q = \frac{-bc+a^2y_0-abx_0}{(a^2+b^2)}.
\]
Mutta nythän
\begin{align*}
PQ &= \sqrt{(x_q-x_0)^2+(y_q-y0)^2} \\
&= \sqrt{\Big(\frac{-ac-aby_0+b^2x_0-(a^2+b^2)x_0}{a^2+b^2}\Big)^2+\Big(\frac{-bc+a^2y_0-abx_0-(a^2+b^2)y_0}{a^2+b^2}\Big)^2} \\
& = \frac{\sqrt{(-ac-aby_0-a^2x_0)^2+(-bc-abx_0-b^2)y_0)^2}}{a^2+b^2} \\
& = \frac{\sqrt{(a(-c-by_0-ax_0))^2+(b(-c-ax_0-by_0))^2}}{a^2+b^2} \\
& = \frac{\sqrt{(a^2+b^2)((-c-by_0-ax_0))^2}}{a^2+b^2} \\
& = \frac{\sqrt{(a^2+b^2)}}{a^2+b^2}\sqrt{(-c-by_0-ax_0))^2} \\
& = \frac{|ax_0+by_0+c|}{\sqrt{a^2+b^2}}.
\end{align*}
\begin{tehtavasivu}

\subsubsection*{Opi perusteet}

\subsubsection*{Hallitse kokonaisuus}

\subsubsection*{Sekalaisia tehtäviä}

TÄHÄN TEHTÄVIÄ SIJOITTAMISTA ODOTTAMAAN

\end{tehtavasivu}
	% kaava pisteen etäisyydelle suorasta

\chapter{Toisen asteen käyrät}
	\section{$\star$ Yleinen toisen asteen tasokäyrä}

\laatikko{
KIRJOITA TÄHÄN LUKUUN

\begin{itemize}
\item Tässä luvussa tarkastellaan lyhyesti yleisesti toisen asteen tasokäyriä, joista ympyrä ja paraabeli on jo käsitelty edellä
\item (kahden muuttujan) toisen asteen yhtälön määritelmä
\item joku tuttu esimerkki, vaikka paraabeli
\item esimerkkeinä yksi piste, kaksi suoraa, tyhjä
\item maininta siitä, että voi tulla myös ellipsi tai hyperbeli,
joista sitten joskus kirjoitetaan liite
\end{itemize}

KIITOS!}

\begin{tehtavasivu}

\subsubsection*{Opi perusteet}

\subsubsection*{Hallitse kokonaisuus}

\subsubsection*{Sekalaisia tehtäviä}

TÄHÄN TEHTÄVIÄ SIJOITTAMISTA ODOTTAMAAN

\end{tehtavasivu}
	% esim. piste, kaksi suoraa, tyhjä
	\section{Ympyrä}

\laatikko{
KIRJOITA TÄHÄN LUKUUN

\begin{itemize}
\item ympyrän määritelmä ja siitä seuraava yhtälö,
origokeskinen ensin
\item muodon $x^2 + y^2 +ax +by +c=0$ täydentäminen neliöksi
ja ympyrän keskipisteen ja säteen selvittäminen siitä
\end{itemize}

KIITOS!}

\begin{kuva}
    kuvaaja.pohja(-3.5, 3.5, -3.5, 3.5, korkeus = 4, nimiX = "$x$", nimiY = "$y$", ruudukko = True)
    kuvaaja.piirraParametri("3*cos(t)", "3*sin(t)", a = 0, b = 2*pi)
    piste((3*cos(0.75), 3*sin(0.75)), "(x, y)", -120)
\end{kuva}

Kuvaan on piirretty käyrä, jonka pisteiden etäisyys origosta on 3. Huomataan, että näin muodostuva käyrä on ympyrä. Piste $(x, y)$ on ympyrällä täsmälleen silloin, jos sen etäisyys origosta on 3. Toisin sanoen täytyy päteä $\sqrt{x^2+y^2}=3$. Kun yhtälön molemmat puolet korotetaan vielä toiseen potenssiin, saadaan $x^2+y^2=9$. Ympyrän yhtälö on siis $x^2+y^2=9$.

\termi{ympyrä}{Ympyrä} muodostuu pisteistä, jotka ovat vakioetäisyydellä jostakin kiinteästä pisteestä, \termi{keskipiste}{keskipisteestä}. Tätä vakioetäisyyttä kutsutaan ympyrän \termi{säde}{säteeksi}.

Johdetaan yhtälö ympyrälle, jonka keskipiste on $(x_0, y_0)$ ja säde $r$. Piste $(x, y)$ on ympyrälllä täsmälleen silloin, jos sen etäisyys pisteestä $(x_0, y_0)$ on $r$. Luvun ?? perusteella pisteiden $(x, y)$ ja $(x_0, y_0)$ välinen etäisyys on $\sqrt{(x-x_0)^2+(y-y_0)^2}$. Tuloksena on siis yhtälö
\[
\sqrt{(x-x_0)^2+(y-y_0)^2}=r.
\]
Koska säde $r$ ei voi olla negatiivinen, voidaan yhtälön molemmat puolet korottaa toiseen potenssiin ja saadaan yhtäpitävä yhtälö
\[
(x-x_0)^2+(y-y_0)^2=r^2.
\]

%Jos erityisesti $(x_{0}, y_{0})= (0, 0)$, eli ympyrän keskipiste on origo, saa yhtälö muodon

%\[
%x^{2}+y^{2} = r^{2}.
%\]

\laatikko{
Jos ympyrän keskipiste on $(x_{0}, y_{0})$ ja säde $r$, ympyrän yhtälö on
\[
(x-x_{0})^{2}+(y-y_{0})^{2} = r^{2}.
\]
}

Jos edellä säde $r$ on nolla, yhtälön toteuttaa vain piste $(x_{0}, y_{0})$. Nollasäteinen ympyrä onkin pelkkä piste.

%\begin{esimerkki}
%Ympyrän keskipiste on $(-4, 1)$ ja säde $5$. Määritä ympyrän yhtälö ja hahmottele ympyrä koordinaatistoon.
%\begin{esimratk}
%Ympyrän yhtälö saadaan käyttämällä edellä annettua kaavaa. Nyt $x_0=-4$, $y_0=1$ ja $r=5$. Ympyrän yhtälöksi saadaan $(x-(-4))^2+(y-1)^2=25$ eli
%\[
%(x+4)^2+(y-1)^2=25.
%\]
%\end{esimratk}
%\begin{esimvast}
%Ympyrän yhtälö on $(x+4)^2+(y-1)^2=25$.
%\end{esimvast}
%\end{esimerkki}

%Tähän kuva ympyrästä.

%Edellisen esimerkin ympyrän yhtälö voidaan kirjoittaa myös toisenlaisessa muodossa.
%\begin{align*}
%(x+4)^2+(y-1)^2&=25 \\
%x^2+8x+16+y^2-2y+1&=25 \\
%x^2+y^2+8x-2y-8&=0.
%\end{align*}

\begin{esimerkki}
Piirrä kuva ympyrästä, jonka yhtälö on
\[
(x-1)^2+(x+1)^2=4.
\]
\begin{esimratk}
Muokataan ympyrän yhtälöä niin, että keskipiste ja säde näkyvät suoraan:
\[
(x-1)^2+(x-(-1))^2=2^2.
\]
Tästä nähdään, että keskipiste on $(1, -1)$ ja säde 2. Nyt kuva on helppo piirtää.

TÄHÄN TARVITAAN KUVA.
\end{esimratk}
\end{esimerkki}

%\begin{esimerkki}
%Ympyrän $\Gamma_{1}$\footnote{sdf} keskipiste on $(3, -4)$ ja säde $\sqrt{2}$, ja ympyrän $\Gamma_{2}$ keskipiste origo ja säde 1. Määritä ympyröiden yhtälöt. Hahmottele ympyrät koordinaatistoon.

%\begin{esimratk}
%Edellisen mukaan ympyrän $\Gamma_{1}$ yhtälö on
%\[
%(x-3)^{2}+(y-(-4))^{2} = (\sqrt{2})^{2}
%\]
%eli sievennettynä
%\[
%(x-3)^{2}+(y+4)^{2} = 2.
%\]
%$\Gamma_{2}$:n yhtälö saadaan vastaavasti:
%\[
%x^{2}+y^{2} = 1.
%\]
%Edelliselle yksikkösäteiselle origokeskiselle ympyrälle on vakiintunut nimitys \emph{yksikköympyrä}.

%\end{esimratk}
%\end{esimerkki}

Aina keskipiste ja säde eivät näy ympyrän yhtälöstä suoraan.
Esimerkiksi yhtälö
\[
x^2+6x+y^2-4y=3
\]
on erään ympyrän yhtälö.
Tämä nähdään täydentämällä summattavat $x^2+6x$ ja $y^2-4y$ neliöiksi.

Neliöksi täydentäminen opittiin kurssissa MAA2, mutta kerrataan se tässä vielä.
Aloitetaan lausekkeesta $x^2+6x$. Se on lähes sama kuin binomin $x+3$ neliö, sillä $(x+3)^2=x^2+6x+9$. Ainoastaan vakiotermit poikkeavat toisistaan ja tämän voi korvata lisäämällä ympyrän yhtälön molemmille puolille luvun $9$: 
\begin{align*}
x^2+6x+y^2-4y &= 3 && \ppalkki +9 \\
(x^2+6x+9)+(y^2-4y) &= 12 && \\
(x+3)^2+(y^2-4y) &= 12.&& 
\end{align*}

Siirrytään sitten tarkastelemaan summattavaa $y^2-4y$. Se on puolestaan melkein binomin $y-2$ neliö, sillä $(y-2)^2=y^2-4y+4$. Binomin neliö saadaan näkyviin lisäämällä yhtälön molemmille puolille luku $4$:
\begin{align*}
(x+3)^2+(y^2-4y) &= 12 && \ppalkki +4\\
(x+3)^2+(y^2-4y+4) &= 16 && \\
(x+3)^2+(y-2)^2 &= 16 && \\
(x+3)^2+(y-2)^2 &= 4^2.&& 
\end{align*}

Nyt huomataan, että kyseessä on $(-3, 2)$-keskisen $4$-säteisen ympyrän yhtälö.

\begin{esimerkki}
Ympyrän yhtälö on $x^2-8x+y^2+5y+3=0$. Määritä ympyrän keskipiste ja säde.
\begin{esimratk}
Yhtälö muuttuu muotoon $x^2-8x+y^2+5y=-3$. Suoritetaan sitten neliöksi täydentäminen:
\begin{align*}
x^2-8x+y^2+5y&=-3 && \ppalkki +16\\
x^2-8x+16+y^2+5y&=-3+16 && \\
(x^2-4)^2+y^2+5y&=13 && \ppalkki +\frac{25}{4}\\
(x^2-4)^2+y^2+5y+\frac{25}{4}&=\frac{77}{4} && \ppalkki +\frac{25}{4}\\
(x^2-4)^2+\left(y+\frac{5}{4}\right)^2&=\frac{102}{4} && \\
(x^2-4)^2+\left(y+\frac{5}{4}\right)^2&=\frac{51}{2} && 
\end{align*}
Nähdään, että keskipiste on $(4, -5/4)$ ja säde $\sqrt{51/2}$.
\end{esimratk}
\begin{esimvast}
Keskipiste on $(4, -5/4)$ ja säde $\sqrt{51/2}$.
\end{esimvast}
\end{esimerkki}

\begin{esimerkki}
Onko yhtälö $x^2-4x+y^2+2y+6=0$ ympyrän yhtälö?
\begin{esimratk}
Suoritetaan neliöksitäydennys:
\begin{align*}
x^2-4x+y^2+2y+6&=0 && \ppalkki -6\\
x^2-4x+y^2+2y&=-6 && \ppalkki +4\\
x^2-4x+4+y^2+2y&=-2 && \\
(x^2-2)^2+y^2+2y&=-2 && \ppalkki +1\\
(x^2-2)^2+y^2+2y+1&=-1 && \\
(x^2-2)^2+(y+1)^2&=-1. &&
\end{align*}
Nyt nähdään, että kyseessä ei voi olla ympyrän yhtälö, sillä säteeksi tulisi $\sqrt{-1}$.
\end{esimratk}
\begin{esimvast}
Kyseessä ei ole ympyrän yhtälö.
\end{esimvast}
\end{esimerkki}

Edellä tehty neliöön korotus voidaan voidaan suorittaa yleisesti muotoa
\[
x^2+ax+y^2+by+c = 0
\]
oleville yhtälöille. Täydentämällä $x^2+ax$ ja $y^2+by$ neliöiksi saadaan
\begin{align*}
x^2+ax+y^2+by+c &= 0 && \ppalkki +\frac{a^2}{4}+\frac{b^2}{4}-c \\
\Big(x^2+ax+\frac{a^2}{4}\Big)+\Big(y^2+by+\frac{b^2}{4}\Big) &= \frac{a^2}{4}+\frac{b^2}{4}-c  \\
\Big(x+\frac{a}{2}\Big)^2+\Big(y+\frac{b}{2}\Big)^2 &= \frac{a^2}{4}+\frac{b^2}{4}-c
\end{align*}
Jos yhtälön oikea puoli eli $\frac{a^2}{4}+\frac{b^2}{4}-c \geq 0$ yhtälö kuvaa $\sqrt{\frac{a^2}{4}+\frac{b^2}{4}-c}$-säteistä $(-\frac{a}{2}, -\frac{b}{2})$-keskistä ympyrää. Jos oikea puoli on nolla, yhtälö kuvaa vastaavaa pistettä. Jos se on negatiivinen, yhtälön vasen puoli on aina positiivinen, joten yhtälön toteuttavia lukupareja ei ole. Yhtälö ei siis kuvaa mitään käyrää.

%\laatikko{Yhtälöt muotoa
%\[
%x^2+ax+y^2+by+c = 0
%\]
%kuvaavat ympyrää, pistettä tai tyhjää joukkoa.}

\begin{tehtavasivu}

\paragraph*{Opi perusteet}

\paragraph*{Hallitse kokonaisuus}

\paragraph*{Sekalaisia tehtäviä}

TÄHÄN TEHTÄVIÄ SIJOITTAMISTA ODOTTAMAAN

\begin{tehtava}
Ympyrän kespiste on $(0, 0)$ ja säde $5$. Muodosta ympyrän yhtälö.
\begin{vastaus}
$x^2+y^2=5$
\end{vastaus}
\end{tehtava}

\begin{tehtava}
Määritä keskipiste ja säde.
\begin{alakohdat}
  \alakohta{$(x-3)^2+(y+7)^2=12$}
	\alakohta{$x^2+y^2=49$}
\end{alakohdat}
\begin{vastaus}
\begin{alakohdat}
	\alakohta{keskpiste $(3, -7)$, säde $2\sqrt{3}$}
	\alakohta{keskipiste $(0, 0)$, säde $7$}
\end{alakohdat}
\end{vastaus}
\end{tehtava}

\begin{tehtava}
Määritä keskipiste ja säde.
\begin{alakohdat}
	\alakohta{$x^2+y^2-10x+16y+72=0$}
	\alakohta{$x^2+y^2+8x-22y+129=0$}
\end{alakohdat}
\begin{vastaus}
\begin{alakohdat}
	\alakohta{keskpiste $(5, -8)$, säde $\sqrt{17}$}
	\alakohta{keskipiste $(-4, 11)$, säde $2\sqrt{2}$}
\end{alakohdat}
\end{vastaus}
\end{tehtava}

\begin{tehtava}
Määritä ympyrän $(x+10)^2+y^2=2$ keskipiste ja säde ja ratkaise ympyrän yhtälöstä $y$. 
\begin{vastaus}
keskipiste $(-10, 0)$, säde $\sqrt{2}$, $y=\pm\sqrt{2-(x+10)^2}$ 
\end{vastaus}
\end{tehtava}

\begin{tehtava}
Ympyrän keskipiste on origo ja säde $3$. Onko piste 
\begin{alakohdat}
	\alakohta{$(10, -2)$}
	\alakohta{$(-3, 0)$}
	\alakohta{$(2, \sqrt{5})$ ympyrän kehällä?}
\end{alakohdat}
\begin{vastaus}
\begin{alakohdat}
	\alakohta{ei!}
	\alakohta{joo!}
	\alakohta{joo!}
\end{alakohdat}
\end{vastaus}
\end{tehtava}

\begin{tehtava}
Määritä $k$ niin, että lauseke $(x-3)^2+(y+3)^2=k$ on
\begin{alakohdat}
	\alakohta{ympyrä}
	\alakohta{$\sqrt{7}$-säteinen ympyrä}
	\alakohta{origon kautta kulkeva ympyrä?}
\end{alakohdat}
\begin{vastaus}
\begin{alakohdat}
	\alakohta{$k>0$}
	\alakohta{$k=7$}
	\alakohta{$k=18$}
\end{alakohdat}
\end{vastaus}
\end{tehtava}

\begin{tehtava}
Tutki, mitä yhtälöiden kuvaajat esittävät.
\begin{alakohdat}
	\alakohta{$x^2+y^2-6x+4y+4=0$}
	\alakohta{$x^2+y^2+14x-6y+10=0$}
\end{alakohdat}
\begin{vastaus}
\begin{alakohdat}
	\alakohta{ympyrä}
	\alakohta{piste}
\end{alakohdat}
\end{vastaus}
\end{tehtava}

\begin{tehtava}
Määritä ympyrän keskipiste ja säde.
\begin{alakohdat}
	\alakohta{$(x+t)^2+(y+u)^2=k, k>0$}
	\alakohta{$(x+2)^2+(y-7)^2=-8$}
\end{alakohdat}
\begin{vastaus}
\begin{alakohdat}
	\alakohta{keskipiste $(-t, -u)$, säde  $\sqrt{k}$}
	\alakohta{ei ole ympyrä}
\end{alakohdat}
\end{vastaus}
\end{tehtava}

\begin{tehtava}
Ympyrä sivuaa $y$-akselia pisteessä $(0, -1)$ ja kulkee pisteen $(3, 2)$ kautta. Mikä on ympyrän yhtälö?
\begin{vastaus}
$(x-3)^2+(y+1)^2=9$
\end{vastaus}
\end{tehtava}

\begin{tehtava}
Ympyrä kulkee pisteiden $(1, 6), (-2, 5)$ ja $(5, 4)$ kautta. Mikä on ympyrän yhtälö?
\begin{vastaus}
$(x-1)^2+(y-1)^2=16$
\end{vastaus}
\end{tehtava}

\begin{tehtava}
Jana, jonka pituus on $t$ liikkuu koordinaatistossa siten, että sen toinen pää on $x$-akselilla ja toinen $y$-akselilla. Mitä käyrää pitkin liikkuu janan keskipiste?
\begin{vastaus}
$x^2+y^2=\frac{1}{4}t^2$
\end{vastaus}
\end{tehtava}

\end{tehtavasivu}
	% ympyrän yhtälö määritelmästä
	% ensin origokeskeinen
	% keskipiste ja säde muissa tapauksissa neliöksi täydentämällä
	\section{Ympyrä ja suora}

\laatikko{
KIRJOITA TÄHÄN LUKUUN

\begin{itemize}
\item ympyrän ja suoran leikkauspisteiden ratkaiseminen
\item ympyrän tangetin määrittäminen sekä kehällä olevan (kohtisuorassa sädettä vastaan) että
sen ulkopuolisen pisteen kautta (kaksi tapaa: pisteen etäisyys suorasta -kaava tai diskriminantti = 0)
\item kahden ympyrän leikkauspisteiden ratkaiseminen yhtälöparilla
\end{itemize}

KIITOS!}

\begin{esimerkki}
Määritä suorien $x+2y-3=0$ ja ympyrän $(x-1)^2+(y+1)^2=4 $ leikkauspisteet.

\begin{esimratk}

Ympyrän ja suoran leikkauspisteet ovat ne pisteet $(x,y)$, jotka ovat sekä suoralla että ympyrällä, eli toteuttavat molempien yhtälöt, eli yhtälöryhmän
$$\left\{    
    \begin{array}{rcl}
        x+2y-3 &=&0 \\
        (x-1)^2+(y+1)^2 &=&4 \\
    \end{array}
    \right.$$
Vaikka yhtälö ei olekaan tutun lineaarinen, myös sitä voi lähestyä sijoitusmenetelmällä. Ratkaistaan ensimmäisestä yhtälöstä $x$ ja saadaan
\[
x = -2y+3
\]
Kun tämä sijoitetaan toiseen yhtälöön ja kerrotaan auki syntyy toisen asteen yhtälö $y$:n suhteen:
\begin{align*}
(-2y+3-1)^2+(y+1)^2=4 \\
(-2y+2)^2+(y+1)^2=4 \\
(-2y)^2-2\cdot 2y\cdot 2 +2^2+y^2+2\cdot y+1^2=4 \\
4y^2-8y+4+y^2+2y+1-4 = 0 \\
5y^2-6y+1 = 0
\end{align*}
Ratkaisukaavalla
\[
y = \frac{-(-6)\pm\sqrt{(-6)^2-4\cdot 5\cdot 1}}{2\cdot5}
\]
eli
\begin{align*}
y = \frac{6\pm\sqrt{16}}{10} \\
y = 1 \vee y = \frac{1}{5}
\end{align*}
Kun nämä $y$:n arvot sijoitetaan suoran yhtälöön saadaan vastaavast $x$:n arvot:
\begin{align*}
x = -2\cdot 1+3 \vee x = -2\cdot\frac{1}{5}+3 \\
x = 1 \vee x = 2\frac{3}{5}
\end{align*}
eli saatiin 2 ratkaisua: $(x,y) = (1,1)$ ja $(x,y) = (2\frac{3}{5},\frac{1}{5})$. Tarkistamalla on hyvä vielä todeta, että pisteet todella ovat leikkauspisteitä.

Tähän kuva tilanteesta.
\begin{esimvast}
Leikkauspisteet ovat $(1,1)$ ja $(2\frac{3}{5},\frac{1}{5})$.
\end{esimvast}

\end{esimratk}
\end{esimerkki}

Esimerkin avulla huomattiin, että suoralla ja ympyrällä voi olla kaksi leikkauspistettä. Suorasta ja ympyrästä riippuen päädytään toisen asteen yhtälöön, jolla on joko 0, 1 tai 2 ratkaisua. Jos leikkauspisteitä on tasan yksi, sanotaan, että suora sivuaa ympyrää tai suora on ympyrän \termi{tangentti}{tangentti}.

Kuva ympyrästä ja kolmesta suorasta; yksi tangentti, yksi leikkaa kahdessa pisteessä ja yksi ei leikkaa ympyrää.

\laatikko{Suoralla ja ympyrällä on nolla, yksi tai kaksi leikkauspisteitä.

Suora on ympyrän \emph{tangentti}, jos suoralla ja ympyrällä on tasan yksi yhteinen piste.
}

\begin{esimerkki}
Määritä $(2,2)$-keskisen 5-säteisen ympyrän pisteen $(-5,3)$ kautta kulkevat tangentit.
\begin{esimratk}
Suora on ympyrän tangentti, jos sillä ja ympyrällä on tasan yksi yhteinen piste. Jos pisteen $(-5,3)$ suora ei ole $y$-akselin suuntainen, se voidaan esittää muodossa
\[
y-3 = k(x-(-5))
\]
eli
\[
y = kx+5k+3.
\]
Ympyrän yhtälö muistaen suoran ja ympyrän leikkauspisteille saadaan siis yhtälöpari
$$\left\{    
    \begin{array}{rcl}
        y &=&kx+5k+3\\
        (x-2)^2+(y-2)^2 &=& 25 \\
    \end{array}
    \right.$$
    
Sijoitetaan ensimmäisen yhtälön lauseke $y$:lle toiseen yhtälöön ja saadaan toisen asteen yhtälö $x$:n suhteen
\begin{align*}
(x-2)^2+(kx+5k+3-2)^2&=25 \\
(x-2)^2+(kx+5k+1)^2&=25 \\
x^2-2\cdot x\cdot 2 +2^2+(kx)^2+kx\cdot 5k+kx&\\
+5k\cdot kx+(5k)^2+5k+kx+5k+1-25& =0 \\
(1+k^2)x^2+(10k^2+2k-4)x+25k^2+10k-20& = 0 \\
\end{align*}
Tällä yhtälöllä on $x$:n suhteen tasan yksi ratkaisu, jos polynomin diskriminantti on nolla, eli
\begin{align*}
(10k^2+2k-4)^2-4*(1+k^2)*(25k^2+10k-20) & = 100k^4+10k^2\cdot 2k+10k^2\cdot (-4) \\
& + 2k\cdot 10k^2+(2k)^2+2k\cdot (-4)+(-4)\cdot 10k^2+(-4)\cdot 2k+(-4)^2
\end{align*}


\end{esimratk}
\end{esimerkki}


\begin{tehtavasivu}

\subsubsection*{Opi perusteet}

\subsubsection*{Hallitse kokonaisuus}
\begin{tehtava}
Määritä yksikköympyrän $x^2+y^2= 1$ pisteeseen $(x_{0},y_{0} )$ piirretyn tangentin normaalimuotoinen yhtälö.
\begin{vastaus}
$x_0x+y_0y=1 $
\end{vastaus}
\end{tehtava}

\subsubsection*{Sekalaisia tehtäviä}

TÄHÄN TEHTÄVIÄ SIJOITTAMISTA ODOTTAMAAN

\end{tehtavasivu}
	% suoran ja ympyrän leikkauspisteet
	% tangentit
	\section{Paraabeli}

\laatikko{
KIRJOITA TÄHÄN LUKUUN

\begin{itemize}
\item käyrän $y = ax^2+by+c$ kuvaaja on paraabeli
%%%terminologia? kuvaaja - funktion kuvaaja - käyrä ovatko samoja eivät?
\item mainitaan geometrinen määritelmä
\item paraabelin yhtälön huippumuoto $y-y_0=a(x-x_0)^2$
\end{itemize}

KIITOS!}

\laatikko{
\termi{paraabeli}{Paraabeli} on tason niiden pisteiden joukko, joiden etäisyys kiinteästä pisteestä, \termi{polttopiste}{polttopisteestä} on sama kuin etäisyys kiinteästä suorasta, \termi{johtosuora}{johtosuorasta}.
}

%%%%%%MAA2, luku 3.1 Toisen asteen polynomifunktio
Kussilla 2 mainittiin, että toisen asteen polynomifunktion kuvaaja on paraabeli. Nämä kuvaajat olivat muotoa $y=ax^2+bx+c$ olevia käyriä, joissa $a$ määräsi paraabelin aukeamissuunnan. Jos $a<0$ paraabeli aukeaa alaspäin ja jos $a>0$ ylöspäin.

\begin{kuva}
    kuvaaja.pohja(-1.5, 3.5, -0.5, 2.5, korkeus = 4, nimiX = "$x$", nimiY = "$y$", ruudukko = True)
    kuvaaja.piirra("0.5*x**2-x+0.25", a = -1.5, b = 3.5, nimi = "$y= 0,5x^2-x+0,25$", kohta = (3.2,2.1), suunta = 135)
\end{kuva}

\begin{kuva}
    kuvaaja.pohja(-1.5, 3.5, -0.5, 2.5, korkeus = 4, nimiX = "$x$", nimiY = "$y$", ruudukko = True)
    kuvaaja.piirra("-0.5*x**2+x+1.75", a = -1.5, b = 3.5, nimi = "$y= -0,5x^2+x+1,75$", kohta = (3.2,-0.5), suunta = 135)
\end{kuva}

\begin{esimerkki}
Määritä pistejoukon yhtälö, jolla on seuraava ominaisuus: Jokainen pistejoukon piste on yhtä etäällä pisteestä $(0, 3)$ ja suorasta $y=-3$
\begin{esimratk}
Pisteen $P=(x, y)$ etäisyys annetusta pisteestä on
\[
\sqrt{(x-0)^2+(y-3)^2}=\sqrt{x^2+(y-3)^2}
\]
Pisteen $P$ etäisyys annetusta suorasta on pisteen ja suoran $y$-koordinaattien erotuksen itseisarvo
\[
|y-(-3)| = |y+3| 
\]
Merkitään nämä etäisyydet yhtäsuuriksi ja ratkaistaan saatu yhtälö $y$:n suhteen.
\begin{align*}
|y+3| & = \sqrt{x^2+(y-3)^2} &&\ppalkki \text{neliöönkorotus, kumpikin puoli $>0$}\\
(y+3)^2  &= x^2+(y-3)^2 \\
y^2+6y+9 &=  x^2+y^2-6y+9\\
12y &= x^2 &&\ppalkki : 12\\
y &= \frac{1}{12}x^2
\end{align*}

Pistejoukon yhtälö on $y=\frac{1}{12}x^2$.

\end{esimratk}
\end{esimerkki}

%%% FIX ME ONKO HUIPPUMUOTOINEN HYVÄ TERMI? %%%%%%%%%%%
\subsection{Paraabelin huippumuotoinen yhtälö}

Kaikki muotoa $y=ax^2+bx+c$ olevat paraabelit voidaan ilmoittaa paraabelin huipun $(x_0, y_0)$ avulla seuraavasti.

\[
y-y_0 = a(x-x_0)^2
\]

%%%%% FIX ME Mikä on tämän alaluvun suhde MAA1-kirjan liitteenä olevaan alalukuun "Toisen asteen polynomin kuvaaja"

\begin{esimerkki}
Mitkä ovat paraabelin $y=3x^2-12x+13$ huipun koordinaatit?
\begin{esimratk}
Paraabelin huipun koordinaatit näkisi suoraan huippumuotoiseksi muutetusta yhtälöstä. Helpompaa lienee kuitenkin muuttaa huippumuoitoinen yhtälö $y$:n suhteen ratkaistuksi ja merkitä yhtälöiden kertoimet samoiksi, jolloin saadaan selville $x_0$ ja $y_0$.

\begin{align*}
y-y_0 &= a(x-x_0)^2 \\
y       &= a(x^2-2x_0x+x_0^2)+y_0\\
y       &= ax^2-2ax_0x+(ax_0^2+y_0) &&\ppalkki a=3\\
y       &= 3x^2-6x_0x+(3x_0^2+y_0)
\end{align*}

Merkitään $x$:n kertoimet ja vakiotermit samoiksi.

\begin{align*}
&\begin{cases}
-6x_0=-12 \\
3x_0^2+y_0 =13
\end{cases}\\
&\begin{cases}
x_0=2 \\
y_0 =1
\end{cases}
\end{align*}

Paraabelin huippu on pisteessä $(2, 1)$.

\end{esimratk}
\end{esimerkki}

\begin{tehtavasivu}

\subsubsection*{Opi perusteet}

\subsubsection*{Hallitse kokonaisuus}

\subsubsection*{Sekalaisia tehtäviä}

TÄHÄN TEHTÄVIÄ SIJOITTAMISTA ODOTTAMAAN

\begin{tehtava}
Millaisella käyrällä ovat ympyröiden keskipisteet, kun ympyrät kulkevat pisteen $(0, 0)$ kautta ja sivuavat suoraa $x=4$?
\begin{vastaus}
%keskipiste (x, y)
% x< 4 
% kulkee origon kautta, joten (0-x)^2+(0-y)^2=r^2
% etäisyys suorasta |x-4| = r
$x=-\frac{1}{8}y^2+2$
\end{vastaus}
\end{tehtava}

\begin{tehtava}
Mikä on sen käyrän yhtälö, jonka kukin piste on yhtä etäällä suorasta $y=0$ ja pisteestä $(-3, 1)$.
\begin{vastaus}
% http://www.wolframalpha.com/input/?i=+sqrt%28%28-3-x%29%5E2+%2B+%281-y%29%5E2%29%3D+abs%280-y%29
$y = \frac{1}{2}x^2+3 x+5$
\end{vastaus}
\end{tehtava}

\begin{tehtava}
%%% ONKO JOHTOSUORA NIIN OLEELLINEN KÄSITE, ETTÄ KÄYTETÄÄN TEHTÄVISSÄ?
%% tehtävänhän voi kirjoittaa ilman tuota termiä
Mikä on sen paraabelin yhtälö, jonka polttopiste on $(2, 3)$ ja johtosuora $y=1$?
\begin{vastaus}
% http://www.wolframalpha.com/input/?i=+sqrt%28%282-x%29%5E2+%2B+%283-y%29%5E2%29%3D+abs%281-y%29
$y = \frac{1}{4}x^2-x+3$
\end{vastaus}
\end{tehtava}

\begin{tehtava}
Mitkä ovat suoran $x+y=6$ ja paraabelin $y=4x^2-3x$ leikkauspisteet?
\begin{vastaus}
% http://www.wolframalpha.com/input/?i=y%3D4x%5E2-3x%2C+x%2By%3D6
$x = -1$, $ y = 7$ tai $x = \frac{3}{2}$, $y = \frac{9}{2}$
\end{vastaus}
\end{tehtava}

%%%%%TÄMÄ JA SEURAAVA PITÄISI SIIRTÄÄ VASEMMALLE/OIKEALLE AUKEAVIEN PARAABELIEN JÄLKEEN?
\begin{tehtava}
Millaisella käyrällä ovat ympyröiden keskipisteet, kun ympyrät kulkevat pisteen $(0, 0)$ kautta ja sivuavat suoraa $x=4$?
\begin{vastaus}
%keskipiste (x, y)
% x< 4 
% kulkee origon kautta, joten (0-x)^2+(0-y)^2=r^2
% etäisyys suorasta |x-4| = r
$x=-\frac{1}{8}y^2+2$
\end{vastaus}
\end{tehtava}

\begin{tehtava}
Kuinka monta leikkauspistettä voi olla paraabelilla ja
\begin{enumerate}[a)]
\item suoralla,
\item ympyrällä,
\item toisella paraabelilla?
\end{enumerate}
\begin{vastaus}
\begin{enumerate}[a)]
\item 0--2
\item 0--4
\item 0--4
\end{enumerate}
\end{vastaus}
\end{tehtava}

\begin{tehtava}
Määritä ne paraabelin $y=x^2-1$ pisteet, jotka ovat yhtä kaukana pisteistä $(4, 4)$ ja $(4, 2)$?
\begin{vastaus}
%suoran y=3 ja paraabelin leikkauspisteet
$x=-2$, $y=3$ ja $x=2$, $y=3$
\end{vastaus}
\end{tehtava}

\begin{tehtava}
Määritä ne paraabelin $y=x^2-1$ pisteet, jotka ovat yhtä kaukana pisteistä $(4, 4)$ ja $(3, 3)$?
\begin{vastaus}
%pisteiden keskipisteen (3,5; 3,5) kautta kulkeva suora y=-x+7
%suoran  ja paraabelin leikkauspistee
% http://www.wolframalpha.com/input/?i=y%3Dx%5E2-1%2C+y%3D-x%2B7
$x = -\frac{1+\sqrt{33}}{2}$,   $y = \frac{15+\sqrt{33}}{2}$ tai $x = -\frac{\sqrt{33}-1}{2}$,   $y = \frac{15-\sqrt{33}}{2}$
\end{vastaus}
\end{tehtava}

\begin{tehtava}
Ratkaise paraabelien $y=5(x-2)^2$  ja $y=-x^2+2x+4$ leikkauspisteet?
\begin{vastaus}
%tulee toisen asteen yhtälö ratkaistavaksi
% http://www.wolframalpha.com/input/?i=y%3D5%28x-2%29%5E2%2C+y%3D-x%5E2%2B2x%2B4
$(1, 5)$ ja $(\frac{8}{3}, \frac{20}{9})$
\end{vastaus}
\end{tehtava}

\begin{tehtava}
%tehtävä helpottuu, koska vakiotermin saa suoraan
Määritä sen ylöspäin aukeavan paraabelin yhtälö, joka kulkee pisteiden $(-5, 6)$, $(0, -4)$ ja $(1, 0)$ kautta.
\begin{vastaus}
% http://www.wolframalpha.com/input/?i=y%3D+x%5E2%2B3x-4+at+x%3D%7B-5%2C+0%2C+1%7D
$y= x^2+3x-4$
\end{vastaus}
\end{tehtava}

\begin{tehtava}
Määritä sen alaspäin aukeavan paraabelin yhtälö, joka kulkee pisteiden $(-5, 6)$, $(0, -4)$ ja $(1, 0)$ kautta.
\begin{vastaus}
% http://www.wolframalpha.com/input/?i=y%3D+x%5E2%2B3x-4+at+x%3D%7B-5%2C+0%2C+1%7D
$y= x^2+3x-4$
\end{vastaus}
\end{tehtava}

\begin{tehtava}
% VAIKEA
Määritä kaksi sellaista paraabelia, että niillä on täsmälleen kolme yhteistä pistettä
\begin{vastaus}
%idea valitaan suora, joka on tangetti kummallekin paraabelille esim. y=x ja kumpikin paraabeli sivuaa suoraa samassa kohtaa
% http://www.wolframalpha.com/input/?i=y%3Dx%5E2-x%2C+x%3D2y%5E2-y
Esimerkiksi $y=x^2-x$ ja  $x=2y^2-y$
\end{vastaus}
\end{tehtava}

\begin{tehtava}
Millä parametrin $a$ arvoilla paraabelin $y=x^2-ax+a$ huippu on $x$-akselilla?
\begin{vastaus}
% esim. neliöksi täydentäminen y=(x-a/2)^2 -a^2/4 +a, josta -a^2/4 +a =0
$a=0$ ja $a=4$
\end{vastaus}
\end{tehtava}

\begin{tehtava}
Määritä vakio $a$ siten, että lausekkeen $2x^2+12x+a$ pienin arvo on 10.
\begin{vastaus}
% esim. neliöksi täydentäminen 2x^2+12x+a= 2((x+3)^2-9+a/2), josta 2(-9+a/2)=10
$a=28$
\end{vastaus}
\end{tehtava}

\begin{tehtava}
Millä vakion $a$ arvolla suora $y=x$ on paraabelin $y=x^2-3x+a$ tangentti?
\begin{vastaus}
% yhtälöllä x = x^2-3x+a pitäisi olla tasan yksi ratkaisu 0 = (x-2)^2-4+a
% http://www.wolframalpha.com/input/?i=y%3D+x%5E2-3x%2B4%2C+y%3Dx
$a=4$
\end{vastaus}
\end{tehtava}

\end{tehtavasivu}


	% merkitys toisen asteen polynomin kuvaajana
	% geometrisen määritelmän maininta
	\section{Paraabelin sovelluksia}

\laatikko{
KIRJOITA TÄHÄN LUKUUN

\begin{itemize}
\item paraabelin huippu on kohdassa $x=-b/2a$, todistus
\item paraabelin yhtälön huippumuoto $y-y_0=a(x-x_0)^2$
\item paraabelin yhtälön ratkaiseminen kolmen pisteen avulla
\item soveltavia tehtäviä, ne iänikuiset holvikaaret jne.
\end{itemize}

KIITOS!}
	% huipun x-koordinaatti on -b/2a
		% todistus liitteeksi -Ville
		% todistus tähän -Niko
	% paraabelin tangentit
	\section{Vasemmalle ja oikealle aukeavat paraabelit}

\laatikko{
KIRJOITA TÄHÄN LUKUUN

\begin{itemize}
\item muotoa $x=ay^2+by+c$ olevat paraabelit aukeavat oikealle tai vasemmalla
\end{itemize}

KIITOS!}

\begin{tehtavasivu}

\subsubsection*{Opi perusteet}

\subsubsection*{Hallitse kokonaisuus}

\subsubsection*{Sekalaisia tehtäviä}

TÄHÄN TEHTÄVIÄ SIJOITTAMISTA ODOTTAMAAN

\end{tehtavasivu}
	% paraabeli x = ay^2  +by + c
	\section{Sekalaista}

	% esimerkkejä ja tehtäviä (erikoisia, vaikeita, yms.)

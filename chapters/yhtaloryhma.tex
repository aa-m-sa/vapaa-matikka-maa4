\section{Lineaariset yhtälöryhmät} % FIXME: siirrä?

\laatikko{
KIRJOITA TÄHÄN LUKUUN

\begin{itemize}
\item mikä yhtälöryhmä on
\item miten ratkaistaan yhtälöpari (sijoitus, yhteenlaskumenetelmä)
\item että ratkaisuja voi olla yksi, nolla tai äärettömän monta
\item miten useamman tuntemattoman yhtälöryhmä ratkaistaan
\end{itemize}

KIITOS!}

%\subsection*{Yhtälöryhmä}

% \termi{yhtälöryhmä}{Yhtälöryhmällä} tarkoitetaan useaa yhtälöä, joiden täytyy
% päteä samanaikaisesti. Englannin kielessä yhtälöryhmä tunnetaankin nimellä
% \textit{simultaneous equations} eli samanaikaiset yhtälöt.
% \footnote{Termi \textit{system of equations} on vähintäänkin yhtä yleinen.}
% Yhtälöryhmän (mahdollisten) ratkaisujen tulee siis toteuttaa kaikki yhtälöt.
% 
% Tässä luvussa käsitellään yhtälöryhmiä, joissa kaikki yhtälöt ovat ensimmäistä
% astetta. Tällaisia yhtälöryhmiä kutsutaan \termi{lineaarinen yhtälöryhmä}{lineaarisiksi yhtälöryhmiksi}.
% Lisäksi myöhemmin kirjassa käsitellään erikoistapauksia toisen asteen yhtälöryhmistä
% ratkaistaessa kahden ympyrän tai ympyrän ja suoran leikkauspisteitä.

\subsection*{Yhtälöpari}

% Yksinkertaisin mielenkiintoinen yhtälöryhmä on
% \termi{lineaarinen yhtälöpari}{lineaarinen yhtälöpari}.
% Lineaarisessa yhtälöparissa on kaksi ensimmäisen asteen yhtälöä.
% Lineaarinen yhtälöpari voidaan esittää monella tapaa. Tässä
% kirjassa käytämme pääasiallisesti ns. normaalimuotoa.
% 
% \[
% \left\{
% \begin{aligned}
% a_1x+b_1y+c_1 &= 0 \\
% a_2x+b_2y+c_2 &= 0
% \end{aligned}
% \right.
% \]
% 
% $a_1, a_2, b_1, b_2, c_1, c_2 \in \R$
% 
% Lineaarisen yhtälöparin ratkaisu on pari $(x, y) \in \R^2$, joka toteuttaa molemmat yhtälöt.

\begin{esimerkki}
Ratkaise yhtälöpari
\[
\left\{
\begin{aligned}
3x-2y&= 1 \\
-x+5y&= 2.
\end{aligned}
\right.
\]
\begin{esimratk}
On löydettävä kaikki ne luvut $x$ ja $y$ jotka toteuttavat molemmat yhtälöt.

Käytetään \termi{sijoitusmenetelmä}{sijoitusmenetelmää}. Ratkaistaan muuttuja $x$ alemmasta yhtälöstä ja sijoitetaan se ylempään yhtälöön. Alemmasta yhtälöstä $-x+5y= 2$ saadaan $x=5y-2$. Sijoitetaan tämä ylempään yhtälöön $3x-2y=1$:
\begin{align*}
3x-2y&=1 && \ppalkki \text{Sijoitetaan $x=5y-2$.} \\
3(5y-2)-2y&=1 && \\
15y-6-2y&=1 && \\
13y&=7 && \\
y&=\frac{13}{7} && \\
\end{align*}
Sijoitetaan $y=13/7$ alempaan yhtälöön, jotta voidaan ratkaista $x$:
\begin{align*}
-x+5y&= 2 && \ppalkki \text{Sijoitetaan $y=\frac{13}{7}$.} \\
-x+\frac{65}{7}&= 2 && \\
-x&= -\frac{51}{7}&& \\
x&= \frac{51}{7}&&
\end{align*}
\end{esimratk}
\begin{esimvast}
Yhtälön ratkaisu on $x= 51/7$, $y=13/7$.
\end{esimvast}
\end{esimerkki}

\begin{esimerkki}
Määritä suorien $-2x-y= 4$ ja $3x-2y=1$ leikkauspiste.
\begin{esimratk}
Saadaan ratkaistavaksi yhtälöpari
\[
\left\{
\begin{aligned}
-2x-y&= 4 \\
3x-2y&= 1.
\end{aligned}
\right.
\]
Käytetään \termi{eliminointimenetelmä}{yhteenlaskumenetelmää}, jolla voidaan eliminoida toinen muuttujista.
\begin{align*}
&\left\{
\begin{aligned}
2x-y&= 4 && \ppalkki \cdot 3\\
3x-2y&= 1. && \ppalkki \cdot (-2)
\end{aligned}
\right. \\
&\left\{
\begin{aligned}
6x-3y&= 12 && \ppalkki \text{Lasketaan yhtälöt yhteen.}\\
-6x+4y&= -2.&&
\end{aligned}
\right.\\
& y&= 10 \\
\end{align*}
Sijoitetaan $y=10$ jompaan kumpaan alkuperäisistä yhtälöistä, jotta saadaan ratkaistua $x$. Käytetään vaikkapa yhtälöä $3x-2y=1$:
\begin{align*}
3x-2y&=1 && \ppalkki \text{Sijoitetaan $x=10$.} \\
3x-20&=1 && \\
3x&=21 && \\
x&=7.
\end{align*}
\end{esimratk}
\begin{esimvast}
Yhtälön ratkaisu on $x=7$, $y=10$.
\end{esimvast}
\end{esimerkki}

Aiemmin on todettu, että normaalimuotoinen ensimmäisen asteen yhtälö voidaan tulkita suorana
$(x, y)$-tasossa. Näin ollen lineaariselle yhtälöparille on geometrinen tulkinta: sen
ratkaisut ovat ne tason pisteet, joissa yhtälöitä vastaavat
suorat leikkaavat. Näitä voi olla
$0$ (suorat ovat yhdensuuntaiset, mutta eivät sama suora),
$1$ (suorat eivät ole yhdensuuntaiset) tai
äärettömän monta (suorat ovat sama suora).

Lineaarisia yhtälöpareja ratkotaan pääasiallisesti kahdella menetelmällä.
Nämä menetelmät ovat \termi{sijoitusmenetelmä}{sijoitusmenetelmä} ja
\termi{yhteenlaskumenetelmä}{yhteenlaskumenetelmä}.


\subsection*{Lineaarinen yhtälöryhmä}

Ratkaistavia yhtälöitä voi olla useampia kuin kaksi, ja tällöin puhutaan \termi{yhtälöryhmä}{yhtälöryhmästä}. Yhtälöpari on siis yhtälöryhmän erikoistapaus.

Jos yhtälöt ovat lisäksi ensimmäistä astetta, on kyseessä \termi{lineaarinen yhtälöryhmä}{lineaarinen yhtälöryhmä}.
Esimerkiksi
\[
\left\{
\begin{aligned}
3x-2y&= -5 \\
5x+6y&= 1 \\
-x+5y&= 0.
\end{aligned}
\right.
\]
on lineaarinen yhtälöryhmä.

% Tässä tarkastellaan lähinnä kolmen yhtälön lineaarisia yhtälöryhmiä. Neljän yhtälön
% lineaarisista yhtälöryhmistä esitetään joitakin helppoja esimerkkejä. Yleisesti ottaen yhtälöryhmiä
% ei ratkaista tällä kurssilla esitetyin keinoin, vaan likimääräisesti tietokoneella käyttäen numeerista 
% matriisilaskentaa, joka ei kuulu lukion oppimäärään.

\subsection*{Lineaarisen yhtälöryhmän ratkaisumenetelmiä}

% tähän menetelmistä

\begin{tehtavasivu}

\subsubsection*{Opi perusteet}

\subsubsection*{Hallitse kokonaisuus}

\subsubsection*{Sekalaisia tehtäviä}

TÄHÄN TEHTÄVIÄ SIJOITTAMISTA ODOTTAMAAN

\begin{tehtava}
    Ratkaise yhtälöpari.
    \begin{align*}
        x+y+1 &= 0 \\
        x+2y+1 &=0
    \end{align*}
    \begin{vastaus}
        $x = -1, \, y = 0$
    \end{vastaus}
\end{tehtava}

\begin{tehtava}
    Ratkaise yhtälöpari.
    \begin{align*}
        2x+5y+1 &= 0 \\
        2x+2y+7 &=0
    \end{align*}
    \begin{vastaus}
        $x = -\frac{11}{2}, \, y = 2$
    \end{vastaus}
\end{tehtava}

\begin{tehtava}
    Ratkaise yhtälöpari. $t \in \R$ on vapaa parametri, joka saa sisältyä vastaukseen.
    \begin{align*}
        x+2y-t-1 &= 0 \\
        x+y+t^2 &=0
    \end{align*}
    \begin{vastaus}
        $x = -2t^2-t-1, \, y = t^2+t+1$
    \end{vastaus}
\end{tehtava}

\begin{tehtava}
    Ratkaise yhtälöryhmä.
	$$\left\{    
    \begin{array}{rcl}
        x+2y+1 &=&0 \\
        x+2z+3 &=&0 \\
        y+2z+5 &=&0
    \end{array}
    \right.$$
    \begin{vastaus}
        $x = 1, \, y = -1, \, z = -2$
    \end{vastaus}
\end{tehtava}

\begin{tehtava}
    Ratkaise yhtälöryhmä.
    \begin{align*}
        x+y+z+8 &= 0 \\
        x+y+6 &=0 \\
        x+z-70 &=0
    \end{align*}
    \begin{vastaus}
        $x = 72, \, y = -78, \, z = -2$
    \end{vastaus}
\end{tehtava}

\begin{tehtava}
    Ratkaise yhtälöryhmä.
    \begin{align*}
        x+y+2z+12 &= 0 \\
        2x+2y+3z+1 &=0 \\
        3x-4 &=0
    \end{align*}
    \begin{vastaus}
        $x = \frac{4}{3}, \, y = \frac{98}{3}, \, z = -23$
    \end{vastaus}
\end{tehtava}

\begin{tehtava}
    Ratkaise yhtälöryhmä.
    \begin{align*}
        2x+3y+5z+8 &= 0 \\
        3x+5y+8z &=0 \\
        x+y-1 &=0
    \end{align*}
    \begin{vastaus}
        $x = -\frac{63}{2}, \, y = \frac{65}{2}, \, z = -\frac{17}{2}$
    \end{vastaus}
\end{tehtava}

\end{tehtavasivu}

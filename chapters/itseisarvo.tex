\section{Itseisarvo}


\laatikko{
KIRJOITA TÄHÄN LUKUUN

\begin{itemize}
\item itseisarvon määritelmä (paloittain määritelty muoto)
\item geometrinen tulkinta lukusuoralla
\item itseisarvoyhtälöistä tyyppiä $|f(x)|=a$, $|f(x)|=|g(x)|$,
$|f(x)|=g(x)$
\end{itemize}

KIITOS!}

Itseisarvo kuvaa lukusuoralla luvun etäisyyttä nollasta. Positiivisen luvun itseisarvo on luku itse ja negatiivisen luvun itseisarvo on luvun vastaluku. Nollan itseisarvo on nolla.


\laatikko{$|x|= \left\{ \begin{array}{rcl}
		x & , kun & x \geq 0 \\
		-x & , kun & x < 0
		\end{array}\right.$}

\begin{esimerkki}
Laske
	\begin{alakohdat}
		\alakohta{$|3-\pi|$}
		\alakohta{$|x-3|$}
	\end{alakohdat}
	\textbf{Ratkaisu}
	\begin{alakohdat}
		\alakohta{Koska $3-\pi\approx-0,14<0$, niin $|3-\pi|=-(3-\pi)=\pi-3$}
		\alakohta{Koska $x-3\geq 0$, kun $x\geq3$, niin\\
			$|x-3|= \left \{ \begin{array}{rcl}
			x-3 & , kun & x\geq3 \\
			3-x & , kun & x<3
			\end{array}\right.$
		}
	\end{alakohdat}
\end{esimerkki}


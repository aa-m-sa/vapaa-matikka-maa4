\section{Ympyrä ja suora}

\laatikko{
KIRJOITA TÄHÄN LUKUUN

\begin{itemize}
\item ympyrän ja suoran leikkauspisteiden ratkaiseminen
\item ympyrän tangetin määrittäminen sekä kehällä olevan (kohtisuorassa sädettä vastaan) että
sen ulkopuolisen pisteen kautta (kaksi tapaa: pisteen etäisyys suorasta -kaava tai diskriminantti = 0)
\item kahden ympyrän leikkauspisteiden ratkaiseminen yhtälöparilla
\end{itemize}

KIITOS!}

\begin{esimerkki}
Määritä suorien $x+2y-3=0$ ja ympyrän $(x-1)^2+(y+1)^2=4 $ leikkauspisteet.

\begin{esimratk}

Ympyrän ja suoran leikkauspisteet ovat ne pisteet $(x,y)$, jotka ovat sekä suoralla että ympyrällä, eli toteuttavat molempien yhtälöt, eli yhtälöryhmän
$$\left\{    
    \begin{array}{rcl}
        x+2y-3 &=&0 \\
        (x-1)^2+(y+1)^2 &=&4 \\
    \end{array}
    \right.$$
Vaikka yhtälö ei olekaan tutun lineaarinen, myös sitä voi lähestyä sijoitusmenetelmällä. Ratkaistaan ensimmäisestä yhtälöstä $x$ ja saadaan
\[
x = -2y+3
\]
Kun tämä sijoitetaan toiseen yhtälöön ja kerrotaan auki syntyy toisen asteen yhtälö $y$:n suhteen:
\begin{align*}
(-2y+3-1)^2+(y+1)^2=4 \\
(-2y+2)^2+(y+1)^2=4 \\
(-2y)^2-2\cdot 2y\cdot 2 +2^2+y^2+2\cdot y+1^2=4 \\
4y^2-8y+4+y^2+2y+1-4 = 0 \\
5y^2-6y+1 = 0
\end{align*}
Ratkaisukaavalla
\[
y = \frac{-(-6)\pm\sqrt{(-6)^2-4\cdot 5\cdot 1}}{2\cdot5}
\]
eli
\begin{align*}
y = \frac{6\pm\sqrt{16}}{10} \\
y = 1 \vee y = \frac{1}{5}
\end{align*}
Kun nämä $y$:n arvot sijoitetaan suoran yhtälöön saadaan vastaavast $x$:n arvot:
\begin{align*}
x = -2\cdot 1+3 \vee x = -2\cdot\frac{1}{5}+3 \\
x = 1 \vee x = 2\frac{3}{5}
\end{align*}
eli saatiin 2 ratkaisua: $(x,y) = (1,1)$ ja $(x,y) = (2\frac{3}{5},\frac{1}{5})$. Tarkistamalla on hyvä vielä todeta, että pisteet todella ovat leikkauspisteitä.

Tähän kuva tilanteesta.
\begin{esimvast}
Leikkauspisteet ovat $(1,1)$ ja $(2\frac{3}{5},\frac{1}{5})$.
\end{esimvast}

\end{esimratk}
\end{esimerkki}

Esimerkin avulla huomattiin, että suoralla ja ympyrällä voi olla kaksi leikkauspistettä. Suorasta ja ympyrästä riippuen päädytään toisen asteen yhtälöön, jolla on joko 0, 1 tai 2 ratkaisua. Jos leikkauspisteitä on tasan yksi, sanotaan, että suora sivuaa ympyrää tai suora on ympyrän \termi{tangentti}{tangentti}.

Kuva ympyrästä ja kolmesta suorasta; yksi tangentti, yksi leikkaa kahdessa pisteessä ja yksi ei leikkaa ympyrää.

\laatikko{Suoralla ja ympyrällä on nolla, yksi tai kaksi leikkauspisteitä.

Suora on ympyrän \emph{tangentti}, jos suoralla ja ympyrällä on tasan yksi yhteinen piste.
}

\begin{esimerkki}
Määritä $(2,2)$-keskisen 5-säteisen ympyrän pisteen $(-5,3)$ kautta kulkevat tangentit.
\begin{esimratk}
Suora on ympyrän tangentti, jos sillä ja ympyrällä on tasan yksi yhteinen piste. Jos pisteen $(-5,3)$ suora ei ole $y$-akselin suuntainen, se voidaan esittää muodossa
\[
y-3 = k(x-(-5))
\]
eli
\[
y = kx+5k+3.
\]
Ympyrän yhtälö muistaen suoran ja ympyrän leikkauspisteille saadaan siis yhtälöpari
$$\left\{    
    \begin{array}{rcl}
        y &=&kx+5k+3\\
        (x-2)^2+(y-2)^2 &=& 25 \\
    \end{array}
    \right.$$
    
Sijoitetaan ensimmäisen yhtälön lauseke $y$:lle toiseen yhtälöön ja saadaan toisen asteen yhtälö $x$:n suhteen
\begin{align*}
(x-2)^2+(kx+5k+3-2)^2&=25 \\
(x-2)^2+(kx+5k+1)^2&=25 \\
x^2-2\cdot x\cdot 2 +2^2+(kx)^2+kx\cdot 5k+kx&\\
+5k\cdot kx+(5k)^2+5k+kx+5k+1-25& =0 \\
(1+k^2)x^2+(10k^2+2k-4)x+25k^2+10k-20& = 0 \\
\end{align*}
Tällä yhtälöllä on $x$:n suhteen tasan yksi ratkaisu, jos polynomin diskriminantti on nolla, eli
\begin{align*}
(10k^2+2k-4)^2-4*(1+k^2)*(25k^2+10k-20) & = & 100k^4+10k^2\cdot 2k+10k^2\cdot (-4) \\
& & + 2k\cdot 10k^2+(2k)^2+2k\cdot (-4)+(-4)\cdot 10k^2+(-4)\cdot 2k+(-4)^2
\end{align*}


\end{esimratk}
\end{esimerkki}


\begin{tehtavasivu}

\subsubsection*{Opi perusteet}

\subsubsection*{Hallitse kokonaisuus}
\begin{tehtava}
	Määritä annetun ympyrän tangentit, jotka kulkevat annetun pisteen kautta.
	\begin{alakohdat}
		\alakohta{Piste $(1,-1)$, ympyrä $(x-3)^2 + (y+1)^2 = 2$}
		\alakohta{Piste $(-4,-2)$, ympyrä $(x+5)^2 + (y+6)^2 = 17$}
		\alakohta{Piste $(3,1)$, ympyrä $(x-4)^2 + y^2 = 1$}
		\alakohta{Piste $(2,4)$, ympyrä $x^2 + (y-3)^2 = 8$}
	\end{alakohdat}
	\begin{vastaus}
		\begin{alakohdat}
			\alakohta{$y=x-2$ ja $y=-x$}
			\alakohta{$y= -1/4 x -3$}
			\alakohta{$y=1$ ja $x=3$}
			\alakohta{Ei ole.}
		\end{alakohdat}
	\end{vastaus}
\end{tehtava}

\begin{tehtava}
Määritä yksikköympyrän $x^2+y^2= 1$ pisteeseen $(x_{0},y_{0} )$ piirretyn tangentin normaalimuotoinen yhtälö.
\begin{vastaus}
$x_0x+y_0y=1 $
\end{vastaus}
\end{tehtava}

\begin{tehtava}
Yksikköympyrälle $x^2+y^2=1$ piirretään tangentit pisteestä $(0,a)$ . 
\begin{alakohdat}
\alakohta{kaksi}
\alakohta{yksi}
\alakohta{nolla?}
\end{alakohdat}
Määritä myös tangenttien yhtälöt.
\begin{vastaus}
\begin{alakohdat}
\alakohta{$a > 1$ tai $a < -1$}
\alakohta{$a = \pm1$}
\alakohta{$ -1 < a < 1 $} 
\end{alakohdat}
Tangenttien yhtälöt ovat $ y = \pm \sqrt{a^2-1}x+a$
\end{vastaus}
\end{tehtava}

\subsubsection*{Sekalaisia tehtäviä}

TÄHÄN TEHTÄVIÄ SIJOITTAMISTA ODOTTAMAAN

\end{tehtavasivu}
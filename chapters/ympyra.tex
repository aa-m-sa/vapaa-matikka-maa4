\section{Ympyrä}

\laatikko{
KIRJOITA TÄHÄN LUKUUN

\begin{itemize}
\item ympyrän määritelmä ja siitä seuraava yhtälö,
origokeskinen ensin
\item muodon $x^2 + y^2 +ax +by +c=0$ täydentäminen neliöksi
ja ympyrän keskipisteen ja säteen selvittäminen siitä
\end{itemize}

KIITOS!}

\emph{Ympyrä} on klassisesti kaikkien kiinteästä pisteestä, \emph{keskipisteestä}, vakioetäisyydellä olevien pisteiden joukko. Tätä vakioetäisyyttä kutsutaan ympyrän \emph{säteeksi}. Jos ympyrän keskipiste on $(x_{0},y_{0})$, ja säde $r$, voidaan ympyrän pisteille muodostaa Pythagoraan lauseen avulla yhtälö.

Määritelmän nojalla mielivaltainen tason piste $(x,y)$ on ympyrällä, jos ja vain jos pisteiden $(x,y)$ ja $(x_{0},y_{0})$ etäisyys on $r$. Pythagoraan lauseella siis

\[
\sqrt{(x-x_{0})^{2}+(y-y_{0})^{2}} = r.
\]

Kun nyt vielä huomioidaan, että $r \geq 0$, voidaan yhtälö yhtä hyvin kirjottaa muodossa

\[
(x-x_{0})^{2}+(y-y_{0})^{2} = r^{2}.
\]
Tapauksessa $r=0$ yhtälön toteuttaa vain pisteestä $(x_{0},y_{0})$ etäisyydellä 0 olevat pisteet, eli piste itse, joten se kuvaa itse keskipistettä. Myös tämä voidaan ajatella surkastuneena, nollasäteisenä ympyränä.

Jos erityisesti $(x_{0},y_{0})= (0,0)$, eli ympyrän keskipiste on origo, saa yhtälö muodon

\[
x^{2}+y^{2} = r^{2}.
\]

\laatikko{
$(x_{0},y_{0})$-keskisen ja r-säteisen ympyrän yhtälö on
\[
(x-x_{0})^{2}+(y-y_{0})^{2} = r^{2}.
\]
}

\begin{esimerkki}
Ympyrän $\Gamma_{1}$\footnote{sdf} keskipiste on $(3,-4)$ ja säde $\sqrt{2}$, ja ympyrän $\Gamma_{2}$ keskipiste origo ja säde 1. Määritä ympyröiden yhtälöt. Hahmottele ympyrät koordinaatistoon.

\begin{esimratk}
Edellisen mukaan ympyrän $\Gamma_{1}$ yhtälö on
\[
(x-3)^{2}+(y-(-4))^{2} = (\sqrt{2})^{2}
\]
eli sievennettynä
\[
(x-3)^{2}+(y+4)^{2} = 2.
\]
$\Gamma_{2}$:n yhtälö saadaan vastaavasti:
\[
x^{2}+y^{2} = 1.
\]
Edelliselle yksikkösäteiselle origokeskiselle ympyrälle on vakiintunut nimitys \emph{yksikköympyrä}.

Tähän kuva ympyröistä.

\end{esimratk}
\end{esimerkki}

Usein myös käänteinen päättely voi olla hyödyllistä. Esimerkiksi yhtälön
\[
(x-1)^2+(x+1)^2=4
\]
nähdään heti kuvaavan 2-säteistä $(1,-1)$ keskistä ympyrää, mutta mitä käyrää kuvaa esimerkiksi yhtälö
\[
x^2+6x+y^2-4y=3.
\]
Yhtälö muistuttaa tavallista ympyrän yhtälöä, auki kerrottuna, ja se voidaankin saattaa normaaliin muotoon täydentämällä binomit $x^2+6x$ ja $y^2-4y$ binomien neliöiksi:
\begin{align*}
x^2+6x+y^2-4y &= 3 && \ppalkki +9+4 \\
(x^2+6x+9)+(y^2-4y+4) &= 3+9+4 \\
(x+3)^2+(y-2)^2 &= 16
\end{align*}
Mutta tämä havaitaan $(-3,2)$-keskisen $\sqrt{16} = 4$ säteisin ympyrän yhtälöksi. Sama päättely voidaan suorittaa myös yleisesti muotoa
\[
x^2+ax+y^2+by+c = 0
\]
olevien yhtälöiden kanssa. Täydentämällä $x^2+ax$ ja $y^2+by$ neliöiksi saadaan
\begin{align*}
x^2+ax+y^2+by+c &= 0 && \ppalkki +\frac{a^2}{4}+\frac{b^2}{4}-c \\
\Big(x^2+ax+\frac{a^2}{4}\Big)+\Big(y^2+by+\frac{b^2}{4}\Big) &= \frac{a^2}{4}+\frac{b^2}{4}-c  \\
\Big(x+\frac{a}{2}\Big)^2+\Big(y+\frac{b}{2}\Big)^2 &= \frac{a^2}{4}+\frac{b^2}{4}-c
\end{align*}
Jos yhtälön oikea puoli eli $\frac{a^2}{4}+\frac{b^2}{4}-c \geq 0$ yhtälö kuvaa $\sqrt{\frac{a^2}{4}+\frac{b^2}{4}-c}$-säteistä $(-\frac{a}{2},-\frac{b}{2})$-keskistä ympyrää. Jos oikea puoli on nolla, yhtälö kuvaa vastaavaa pistettä. Jos se on negatiivinen, yhtälön vasen puoli on aina positiivinen, joten yhtälön toteuttavia lukupareja ei ole, ts. yhtälö kuvaa tyhjää joukkoa.

\laatikko{Yhtälöt muotoa
\[
x^2+ax+y^2+by+c = 0
\]
kuvaavat ympyrää, pistettä tai tyhjää joukkoa.}




\begin{tehtavasivu}

\paragraph*{Opi perusteet}

\paragraph*{Hallitse kokonaisuus}

\paragraph*{Sekalaisia tehtäviä}

TÄHÄN TEHTÄVIÄ SIJOITTAMISTA ODOTTAMAAN

\begin{tehtava}
Ympyrän kespiste on $(0,0)$ ja säde $5$. Muodosta ympyrän yhtälö.
\begin{vastaus}
$x^2+y^2=5$
\end{vastaus}
\end{tehtava}

\begin{tehtava}
Märitä keskipiste ja säde.
\begin{alakohdat}
  \alakohta{$(x-3)^2+(y+7)^2=12$}
	\alakohta{$x^2+y^2=49$}
\end{alakohdat}
\begin{vastaus}
\begin{alakohdat}
	\alakohta{keskpiste $(3,-7)$, säde $2\sqrt{3}$}
	\alakohta{keskipiste $(0,0)$, säde $7$}
\end{alakohdat}
\end{vastaus}
\end{tehtava}

\begin{tehtava}
Märitä keskipiste ja säde.
\begin{alakohdat}
	\alakohta{$x^2+y^2-10x+16y+72=0$}
	\alakohta{$x^2+y^2+8x-22y+129=0$}
\end{alakohdat}
\begin{vastaus}
\begin{alakohdat}
	\alakohta{keskpiste $(5,-8)$, säde $\sqrt{17}$}
	\alakohta{keskipiste $(-4,11)$, säde $2\sqrt{2}$}
\end{alakohdat}
\end{vastaus}
\end{tehtava}

\begin{tehtava}
Määritä ympyrän $(x+10)^2+y^2=2$ keskipiste ja säde ja ratkaise ympyrän yhtälöstä $y$. 
\begin{vastaus}
keskipiste $(-10,0)$, säde $\sqrt{2}$, $y=\pm\sqrt{2-(x+10)^2}$ 
\end{vastaus}
\end{tehtava}

\begin{tehtava}
Ympyrän keskipiste on origo ja säde $3$. Onko piste 
\begin{alakohdat}
	\alakohta{$(10,-2)$}
	\alakohta{$(-3,0)$}
	\alakohta{$(2,\sqrt{5})$ ympyrän kehällä?}
\end{alakohdat}
\begin{vastaus}
\begin{alakohdat}
	\alakohta{ei!}
	\alakohta{joo!}
	\alakohta{joo!}
\end{alakohdat}
\end{vastaus}
\end{tehtava}

\begin{tehtava}
Määritä $k$ niin, että lauseke $(x-3)^2+(y+3)^2=k$ on
\begin{alakohdat}
	\alakohta{ympyrä}
	\alakohta{$\sqrt{7}$-säteinen ympyrä}
	\alakohta{origon kautta kulkeva ympyrä?}
\end{alakohdat}
\begin{vastaus}
\begin{alakohdat}
	\alakohta{$k>0$}
	\alakohta{$k=7$}
	\alakohta{$k=18$}
\end{alakohdat}
\end{vastaus}
\end{tehtava}

\begin{tehtava}
Tutki, mitä yhtälöiden kuvaajat esittävät.
\begin{alakohdat}
	\alakohta{$x^2+y^2-6x+4y+4=0$}
	\alakohta{$x^2+y^2+14x-6y+10=0$}
\end{alakohdat}
\begin{vastaus}
\begin{alakohdat}
	\alakohta{ympyrä}
	\alakohta{piste}
\end{alakohdat}
\end{vastaus}
\end{tehtava}

\begin{tehtava}
Määritä ympyrän keskipiste ja säde.
\begin{alakohdat}
	\alakohta{$(x+t)^2+(y+u)^2=k, k>0$}
	\alakohta{$(x+2)^2+(y-7)^2=-8$}
\end{alakohdat}
\begin{vastaus}
\begin{alakohdat}
	\alakohta{keskipiste $(-t,-u)$, säde  $\sqrt{k}$}
	\alakohta{ei ole ympyrä}
\end{alakohdat}
\end{vastaus}
\end{tehtava}

\begin{tehtava}
Ympyrä sivuaa $y$-akselia pisteessä $(0,-1)$ ja kulkee pisteen $(3,2)$ kautta. Mikä on ympyrän yhtälö?
\begin{vastaus}
$(x-3)^2+(y+1)^2=9$
\end{vastaus}
\end{tehtava}

\begin{tehtava}
Ympyrä kulkee pisteiden $(1,6), (-2,5)$ ja $(5,4)$ kautta. Mikä on ympyrän yhtälö?
\begin{vastaus}
$(x-1)^2+(y-1)^2=16$
\end{vastaus}
\end{tehtava}

\begin{tehtava}
Jana, jonka pituus on $t$ liikkuu koordinaatistossa siten, että sen toinen pää on $x$-akselilla ja toinen $y$-akselilla. Mitä käyrää pitkin liikkuu janan keskipiste?
\begin{vastaus}
$x^2+y^2=\frac{1}{4}t^2$
\end{vastaus}
\end{tehtava}

\end{tehtavasivu}
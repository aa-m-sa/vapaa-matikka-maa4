\section{Vasemmalle ja oikealle aukeavat paraabelit}

\laatikko{
KIRJOITA TÄHÄN LUKUUN

\begin{itemize}
\item muotoa $x=ay^2+by+c$ olevat paraabelit aukeavat oikealle tai vasemmalle
\end{itemize}

KIITOS!}

%Määritelmä

Edellisissä kappaleissa käsiteltiin muotoa $y = ax^2 + b + c$ olevia paraabeleita, jotka ovat symmetrisiä pystyakselin suhteen (eli akselin oikea puoli on vasemman puolen peilikuva). Vaihtamalla paraabelin yhtälössä $x$- ja $y$-muuttujat keskenään saadaan yhtälö
\[x=ay^2+by+c.\]
Sen kuvaaja on joko oikealle tai vasemmalle aukeava paraabeli, jonka symmetria-akseli on vaakasuora. Kun $a>0$ paraabeli aukeaa oikealle, ja kun $a < 0$ paraabeli aukeaa vasemmalle.

%FIXME Tähän kuva, vasemmalle ja oikealle aukeava paraabeli

%FIXME Mikä on vakiintunut suomenkielinen terminologia? 
%Tässä ja alla käytetty samaa "oikealle tai vasemmalla aukeava (...)" -rakennetta kuin otsikossa

%Laatikko x=ay^2+by+c
\laatikko{Yhtälö $x=ay^2+by+c$ määrittelee paraabelin, joka aukeaa oikealle tai vasemmalle.}

%FIXME Yksinkertainen tehtäväesimerkki

Vasemmalle ja oikealle aukeaville paraabeleille pätevät vastaavat tulokset kuin aikaisemmin käsitellyille pystysuorille paraabeleille. Esimerkiksi vaakasuuntaisen paraabelin huippu sijaitsee aina pisteessä

\[y = \frac{-b}{2a}.\]

%Huippumuoto käännetylle paraabelille

Myös $x=ay^2+by+c$ voidaan kirjoittaa huippumuotoisena yhtälönä:

\laatikko{Oikealle tai vasemmalle aukeavan paraabelin huippumuotoinen yhtälö on
\[
x-x_0 = a(y-y_0)^2.
\]}

%FIXME Tähän tehtäväesimerkki ylläolevasta

%Muita esimerkkejä?

%FIXME Paraabelin yleinen muoto (johtosuora ax +by +c = 0) (kartioleikkaukset??) liitteisiin?

\begin{tehtavasivu}

\subsubsection*{Opi perusteet}

\begin{tehtava}
    Mihin suuntaan aukeaa paraabeli, jonka yhtälö on
    \begin{alakohdat}
        \alakohta{$x = -2y^2 + y + 1$}
        \alakohta{$x = y^2 - 3y -2$}
        \alakohta{$x = 4y^2 - 1y +3$}
    \end{alakohdat}
    \begin{vastaus}
          \begin{alakohdat}
          \alakohta{Vasemmalle.}
          \alakohta{Oikealle.}
          \alakohta{Oikealle.}
          \end{alakohdat}
    \end{vastaus}
\end{tehtava}

\begin{tehtava}
    Hahmottele edellisen tehtävän paraabelit ja laske niiden huippujen koordinaatit.
    \begin{vastaus}
          Huiput sijaitsevat pisteissä $\left(\frac{9}{8}, \frac{1}{4}\right)$, $\left(\frac{17}{4},\frac{3}{2}\right)$ ja $\left(\frac{47}{16},\frac{1}{8}\right).$
    \end{vastaus}
\end{tehtava}

\subsubsection*{Hallitse kokonaisuus}

\subsubsection*{Sekalaisia tehtäviä}

TÄHÄN TEHTÄVIÄ SIJOITTAMISTA ODOTTAMAAN


\end{tehtavasivu}
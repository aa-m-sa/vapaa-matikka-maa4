\section{Lisämateriaalia: Kolmioepäyhtälö}

\begin{esimerkki}
Määrittele, mitä tarkoitetaan reaaliluvun $x$ itseisarvolla. \\
Todista, että seuraavat epäyhtälöt ovat voimassa kaikille reaaliluvuille $x$ ja $y$:
\begin{alakohdat}
\alakohta{$x \leq |x|$}
\alakohta{$x+y \leq |x|+|y|$}
\alakohta{$|x+y| \leq |x|+|y|$}
\alakohta{$||x|-|y|| \leq |x|+|y|$}
\end{alakohdat}
(YO syksy08/14)
\begin{esimratk}
Itseisarvo määriteltiin kirjan alussa seuraavasti:
\[
|x|=\begin{cases}
	x, & \kun x \geq 0 \\
	-x, & \kun x < 0
\end{cases} 
\]
\begin{alakohdat}
\alakohta{Kun $x \geq 0$, väite pätee selvästi muodossa $x \leq x$. Koska $|x| \geq 0$
kaikilla reaaliluvuilla $x$, niin väite pätee muodossa $x<0 \leq |x|$, kun $x<0$}
\alakohta{Käytetään edellistä kohtaa: sen mukaan $x+y \leq |x|+y \leq |x|+|y|$.}
\alakohta{Edellisen kohdan nojalla $-(x+y)=-x+(-y) \leq |-x|+|-y|=|x|+|y|$. Koska
$x+y \leq |x|+|y|$ ja $-(x+y) \leq |x|+|y|$, niin myös $|x+y| \leq |x|+|y|$.}
\alakohta{Soveltamalla edellistä kohtaa saadaan suoraan $||x|-|y|| = ||x|+(-|y|)| \leq
||x||+|-|y|| = |x|+|y|$.}
\end{alakohdat}
\end{esimratk}
\end{esimerkki}

Edellä todistetuista väitteistä c)-kohta kulkee nimellä kolmioepäyhtälö, ja se on erittäin
kätevä työkalu edistyneessä matematiikassa.

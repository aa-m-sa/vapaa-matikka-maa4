\section{Paraabelin sovelluksia}

\laatikko{
KIRJOITA TÄHÄN LUKUUN

\begin{itemize}
\item paraabelin huippu on kohdassa $x=-b/2a$, todistus
\item paraabelin yhtälön huippumuoto $y-y_0=a(x-x_0)^2$
\item paraabelin yhtälön ratkaiseminen kolmen pisteen avulla
\item soveltavia tehtäviä, ne iänikuiset holvikaaret jne.
\end{itemize}

KIITOS!}

\begin{tehtavasivu}

\subsubsection*{Opi perusteet}

\begin{tehtava}
Arkkitehti Guggenheim suunnittelee uuteen taidemuseoon kahta paraabelin muotoista holvikaarta. Holvikaarten leveys on 5 metriä ja korkeus 7 metriä, ne ovat puolen metrin päässä toisistaan ja ne ovat sijoitettu symmetrisesti (y-akseliin nähden) julkisivulle. Määritä holvikaarten yhtälöt.
\begin{vastaus}
%holvikaaret puoli metriä toisistaan, siis etäisyys y-akselista 0,25m.
%Tässä ensimmäinen piste, toinen piste on 5,25m päässä, ja kolmas on (2,75; 7). %Riittää määrittää vain yksi paraabeli, ja toisen saa x_0:n vastaluvusta.
$y-7 = -\frac{6,25}{7}(x - 2,75)^2$ ja $y-7 = -\frac{6,25}{7}(x + 2,75)^2$
\end{vastaus}
\end{tehtava}



\subsubsection*{Hallitse kokonaisuus}

\subsubsection*{Sekalaisia tehtäviä}

TÄHÄN TEHTÄVIÄ SIJOITTAMISTA ODOTTAMAAN

\end{tehtavasivu}
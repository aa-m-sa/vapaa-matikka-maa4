\section{Suoran yhtälö}

\laatikko{
KIRJOITA TÄHÄN LUKUUN

\begin{itemize}
\item suoran yhtälö muodossa $y=kx+b$
\item kulmakertoimen ja vakiotermin merkitys
\item pysty- ja vaakasuoran suoran yhtälö
\item suorien leikkauspisteen/suoran ja $x$-akselin leikkauspisteen ratkaiseminen
\end{itemize}

KIITOS!}

Alla on kolme kuvaa, joihin on seuraavia yhtälöitä vastaavat kuvaajat:
\begin{align*}
y & =x-1 \\
y & =2x \\
\text{ja} \quad y & =-x+2.
\end{align*}
Kuten huomataan, kaikki kolme kuvaajaa ovat suoria.

\begin{kuva}
    kuvaaja.pohja(-3.5, 3.5, -3.5, 3.5, korkeus = 4, nimiX = "$x$", nimiY = "$y$", ruudukko = True)
    kuvaaja.piirra("x-1", nimi = "$y=x-1$")
    kuvaaja.piirra("2*x", nimi = "$y=2x$", suunta = 45)
    kuvaaja.piirra("-x+2", nimi = "$y=-x+2$", kohta = -0.6, suunta = -135)
\end{kuva}

\laatikko{
Yhtälön
\[
y=kx+b
\]
määräämä kuvaaja on suora. Lukua $k$ nimitetään suoran \termi{kulmakerroin}{kulmakertoimeksi} ja lukua $b$ \termi{vakiotermi}{vakiotermiksi}.
}


%%%%%%FIXME Pitäisikö tässä olla kolme pistettä, kun käytännössä aika usein kolmas olisi hyvä, koska huolimattomuusvirhe

\begin{esimerkki} Piirretään koordinaatistoon yhtälön $y=2x-3$ kuvaaja. Koska kyseessä on suoran yhtälö, riittää löytää kaksi pistettä, joiden kautta suora kulkee.
Valitaan esimerkiksi pisteet, joiden $x$-koordinaatit ovat $0$ ja $2$. Ensimmäisen $y$-koordinaatti on
\[
y=2\cdot 0-3=-3
\]
ja toisen
\[
y=2\cdot 2-3=4-3=1.
\]
Suoran pisteet ovat siis $(0, -3)$ ja $(2, 1)$. Piirretään nämä koordinaatistoon ja vedetään niiden kautta suora.

\begin{kuva}
    kuvaaja.pohja(-1.5, 3.5, -3.5, 3.5, korkeus = 4, nimiX = "$x$", nimiY = "$y$", ruudukko = True)
    kuvaaja.piirra("2*x-3", nimi = "$y=2x-3$")
    piste((0, -3))
    piste((2, 1))
\end{kuva}

\end{esimerkki}

\subsubsection*{Kulmakertoimen tulkinta}

Suoran kulmakerroin kertoo, miten jyrkästi suora nousee tai laskee. Tarkastellaan alla olevaa suoraa, jonka yhtälö on $y=2x$.

[KUVASTA PUUTTUU KAKSI PISTETTÄ JA KOLMIOT]

\begin{kuva}
    kuvaaja.pohja(-1, 3, -1, 5, korkeus = 4, nimiX = "$x$", nimiY = "$y$", ruudukko = True)
    kuvaaja.piirra("2*x", nimi = "$y=2x$")
    piste((1, 2))
    piste((2, 4))
\end{kuva}


Valitaan suoralta kaksi pistettä,
$A=(1, 2)$ ja $B=(2, 4)$. Siirryttäessä pisteestä $A$ pisteeseen $B$ $x$-koordinaatin arvo kasvaa yhdellä ja $y$-koordinaatin arvo kahdella. Saadaan suhde
\[
\frac{\text{$y$-koordinaatin muutos}}{\text{$x$-koordinaatin muutos}}=\frac{2}{1}=2.
\]
Jos nyt valitaan suoralta jotkin toiset pisteet, esimerkiksi $D=(-1, -2)$ ja $E=(5, 10)$, voidaan laskea samalla tavalla
\[
\frac{\text{$y$-koordinaatin muutos}}{\text{$x$-koordinaatin muutos}}=\frac{10-(-2)}{5-(-1)}=\frac{12}{6}=2.
\]
Huomataan, että yllä laskettu suhde on aina sama pisteistä riippumatta. Tämä johtuu siitä, että kuvan kolmiot $ABC$ ja $DEF$ ovat yhdenmuotoisia.
Suhde on lisäksi sama kuin suoran yhtälössä esiintyvä kulmakerroin.

\begin{esimerkki} Määritä alla olevien suorien kulmakertoimet.

%%% pitäisikö pisteet olla piirrettyinä kuvaan?

\begin{kuva}
    kuvaaja.pohja(-2.5, 3, -1, 5, korkeus = 4, nimiX = "$x$", nimiY = "$y$", ruudukko = True)
    kuvaaja.piirra("3*x+1")
    kuvaaja.piirra("-0.5*x+3")
    piste((0, 1))
    piste((1, 4))
    piste((-2, 4))
    piste((2, 2))
\end{kuva}

\begin{esimratk} Valitaan suorilta kaksi mielivaltaista pistettä ja lasketaan $y$-koordinaatin muutoksen suhde $x$-koordinaatin muutokseen.
Ensimmäiseltä suoralta valitaan vaikkapa pisteet $(0, 2)$ ja $(1, 4)$. Nyt kulmakertoimeksi tulee
\[
\frac{\text{$y$-koordinaatin muutos}}{\text{$x$-koordinaatin muutos}}=\frac{4-1}{1-0}=\frac{3}{1}=3.
\]
Toiselta suoralta valitaan pisteet $(-2, 4)$ ja $(2, 2)$. Nyt täytyy olla tarkkana etumerkkien kanssa:
\[
\frac{\text{$y$-koordinaatin muutos}}{\text{$x$-koordinaatin muutos}}=\frac{2-4}{2-(-2)}=\frac{-2}{4}=-\frac{1}{2}.
\]
\end{esimratk}

\begin{esimvast}
Ensimmäisen suoran kulmakerroin on $3$ ja toisen $-\frac{1}{2}$.
\end{esimvast}
\end{esimerkki}

Kulmakerroin kertoo suoran suunnasta: mitä suurempi kulmakerroin, sitä jyrkemmin suora nousee koordinaatistossa oikealle päin.
Jos kulmakerroin on negatiivinen, suora on laskeva. Vaakasuoran suoran kulmakerroin on 0.

\subsubsection*{Vakiotermin tulkinta}

Yllä nähtiin, että suora $y=2x$ kulkee origon kautta. Seuraavassa kuvassa on suoran $y=2x+1$ kuvaaja. Se saadaan nostamalla suoraa $y=2x$ yhden yksikön verran ylöspäin.

\begin{kuva}
    kuvaaja.pohja(-1, 3, -1, 5, korkeus = 4, nimiX = "$x$", nimiY = "$y$", ruudukko = True)
    kuvaaja.piirra("2*x+1", nimi = "$y=2x+1$", suunta = 45)
    vari("lightgray")
    kuvaaja.piirra("2*x")
\end{kuva}

Tarkastellaan suoralla $y=kx+b$ olevaa pistettä, jonka $x$-koordinaatti on 0.
Tämä piste sijaitsee $y$-akselilla. Toisin sanoen se on suoran ja $y$-akselin leikkauspisteessä.
Sen $y$-koordinaatti saadaan laskemalla
\[
y=k\cdot 0+b=b.
\]
Pisteen $y$-koordinaatti on siis $b$, eli suoran yhtälön vakiotermi. Vakiotermi siis ilmaisee, missä kohtaa suora leikkaa $y$-akselin.
Alla on esimerkkejä erilaisista vakiotermeistä.

%%%% Nyt on samat kulmakertoimet kuin aiemmissa esimerkeissä.
%%%% Jos kulmakerroin olisi esim. 0.5, niin nimet saisi oikealle puolelle nätisti

\begin{kuva}
    kuvaaja.pohja(-2, 3, -1, 5, korkeus = 4, nimiX = "$x$", nimiY = "$y$", ruudukko = True)
    kuvaaja.piirra("2*x+1", nimi = "$y=2x+1$", suunta = 45)
    kuvaaja.piirra("2*x+3", nimi = "$y=2x+3$", kohta = -2, suunta = -135)
    kuvaaja.piirra("2*x-1", nimi = "$y=2x-1$", suunta = -45)
    kuvaaja.piirra("2*x-5", nimi = "$y=2x-5$", suunta = 0)
\end{kuva}

\subsubsection*{Suoran yhtälön määrittäminen}

Suoran yhtälö voidaan määrittää kuvasta laskemalla suoran kulmakerroin sekä vakiotermi.

\begin{esimerkki} Mikä on alla olevan kuvan suoran yhtälö?

\begin{kuva}
    kuvaaja.pohja(-1, 4.5, -0.5, 3.5, korkeus = 4, nimiX = "$x$", nimiY = "$y$", ruudukko = True)
    kuvaaja.piirra("0.5*x+1")
    piste((0, 1), "(0, 1)", 135)
    piste((4, 3), "(4, 3)", 135)
\end{kuva}


\begin{esimratk}
Aloitetaan määrittämällä kulmakerroin. Valitaan suoralta pisteet $(0, 1)$ ja $(4, 3)$. Kulmakertoimeksi tulee
\[
k=\frac{3-1}{4-0}=\frac{2}{4}=\frac{1}{2}\text{.}
\]
\end{esimratk}
Vakiotermi saadaan kohdasta, jossa suora leikkaa $y$-akselin. Tuossa kohdassa $y$-koordinaatti on 1. Vakiotermi on siis $b=1$.
\begin{esimvast}
Suoran yhtälö on $y=\frac{1}{2}x+1$.
\end{esimvast}
\end{esimerkki}

Kun kulmakerroin on 0, suora on vaakasuora. Sen yhtälö on siis muotoa
\laatikko[Vaakasuora suora]{
\[
y=b.
\]
}
Toisaalta pystysuoralla suoralla ei ole kulmakerrointa lainkaan. Sen yhtälöä ei voi ilmaista muodossa $y=\dots$, vaan sillä on yhtälö
\laatikko[Pystysuora suora]{
\[
x=a.
\]
}
Tässä $a$ on sen pisteen $x$-koordinaatti, jossa suora leikkaa $x$-akselin.

\begin{tehtavasivu}

\subsubsection*{Opi perusteet}

\begin{tehtava}
Määritä suoran $y=3x+1$ ja $x$-akselin leikkauspiste.
\begin{vastaus}
$x=-\frac{1}{3}$
\end{vastaus}
\end{tehtava}

\begin{tehtava}
Ratkaise suorien $y=-5x+3$ ja $y=2x-17$ leikkauspiste.
\begin{vastaus}
% http://www.wolframalpha.com/input/?i=y%3D-5x%2B3%2C+y%3D2x-17
$(\frac{20}{7}, -\frac{79}{70})$
\end{vastaus}
\end{tehtava}

\begin{tehtava}
Mikä on $x$-akselin suuntaisen suoran, joka kulkee pisteen $(1, 3)$ kautta, yhtälö?
\begin{vastaus}
$y=3$
\end{vastaus}
\end{tehtava}

\begin{tehtava}
Mikä on suoran $y=3,14x-10$ yhtälön
\begin{enumerate}[a)]
\item vakiotermi,
\item kulmakerroin?
\end{enumerate}
\begin{vastaus}
a)$-10$ b) $3,14$
\end{vastaus}
\end{tehtava}

\begin{tehtava}
Suora kulkee pisteiden $(2, 1)$ ja $(5, 9)$ kautta. Määritä suoran kulmakerroin.
\begin{vastaus}
Kulmakerroin on $\frac{8}{3}$
\end{vastaus}
\end{tehtava}

\begin{tehtava}
Piirrä suora $y=9x-1$.
\begin{vastaus}
puuttuu
\end{vastaus}
\end{tehtava}

\subsubsection*{Hallitse kokonaisuus}

\begin{tehtava}
Ratkaise suorien $y=-x+2$ ja $y=2x-4$ leikkauspiste.
\begin{vastaus}
$(2, 0)$
\end{vastaus}
\end{tehtava}

\begin{tehtava}
Määritä
\begin{enumerate}[a)]
\item $x$-akselin suuntaisen suoran,
\item $y$-akselin suuntaisen suoran kulmakerroin?
\end{enumerate}
\begin{vastaus}
a) $0$ b) ei määritelty %(ääretön)
\end{vastaus}
\end{tehtava}

\begin{tehtava}
Piirrä suora $y=-2x+3$.
\begin{vastaus}
puuttuu
\end{vastaus}
\end{tehtava}

\begin{tehtava}
Määritä suoran $\frac{y}{2}=\frac{x}{2}+2$ ja $x$-akselin leikkauspiste.
\begin{vastaus}
$(-4, 0)$
\end{vastaus}
\end{tehtava}

\begin{tehtava}
Ratkaise suoran $6=-60x+600y$ ja $x$-akselin leikkauspiste.
\begin{vastaus}
$x=-\frac{1}{10}$
\end{vastaus}
\end{tehtava}

\begin{tehtava}
Missä pistessä suora $y=\frac{16x}{25}+\frac{36}{49}$
\begin{enumerate}[a)]
\item leikkaa $x$-akselin,
\item leikkaa $y$-akselin?
\end{enumerate}
\begin{vastaus}
a)$(-\frac{225}{196}, 0)$ b) $(0, \frac{36}{49})$
\end{vastaus}
\end{tehtava}

\begin{tehtava}
Ratkaise suorien $y=-\frac{2}{5}$ ja $3y=18x+20$ leikkauspiste.
\begin{vastaus}
$(-\frac{53}{45}, -\frac{2}{5})$
\end{vastaus}
\end{tehtava}

\begin{tehtava}
Ratkaise suoran $16y-9x=-5y-11x+27$ ja $x$-akselin leikkauspiste.
\begin{vastaus}
$(11, 0)$
\end{vastaus}
\end{tehtava}

\subsubsection*{Sekalaisia tehtäviä}

LAITA TEHTÄVÄT TÄHÄN, JOS ET OLE VARMA VAIKEUSASTEESTA TAI TEHTÄVÄ
EI TÄLLÄ HETKELLÄ SOVI MUKAAN

\end{tehtavasivu}
